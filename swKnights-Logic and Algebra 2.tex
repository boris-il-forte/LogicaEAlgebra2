%% LyX 2.0.6 created this file.  For more info, see http://www.lyx.org/.
%% Do not edit unless you really know what you are doing.
\documentclass[11pt,twoside,english,italian]{report}
\renewcommand{\ttdefault}{mathpazo}
\usepackage[T1]{fontenc}
\usepackage[utf8]{inputenc}
\usepackage[a4paper]{geometry}
\geometry{verbose}
\setcounter{secnumdepth}{3}
\setcounter{tocdepth}{3}
\usepackage{fancybox}
\usepackage{calc}
\usepackage{amsmath}
\usepackage{amssymb}
\PassOptionsToPackage{normalem}{ulem}
\usepackage{ulem}

\makeatletter

%%%%%%%%%%%%%%%%%%%%%%%%%%%%%% LyX specific LaTeX commands.
%% Because html converters don't know tabularnewline
\providecommand{\tabularnewline}{\\}

%%%%%%%%%%%%%%%%%%%%%%%%%%%%%% User specified LaTeX commands.

\newlength\tindent
\setlength{\tindent}{\parindent}
\setlength{\parindent}{0pt}
\renewcommand{\indent}{\hspace*{\tindent}}

\title{Logica e Algebra 2}
 
\usepackage{tikz}

\usepackage{color}   %May be necessary if you want to color links
\usepackage{hyperref}
\hypersetup{
    colorlinks=true, %set true if you want colored links
    linktoc=all,     %set to all if you want both sections and subsections linked
    linkcolor=black,  %choose some color if you want links to stand out
}

\usepackage{booktabs}
\usepackage{listings}
\lstset{columns=fullflexible}

\usepackage{wasysym}

\usepackage{amsthm}
\newtheoremstyle{note} % name
{\topsep} 	% Space above
{\topsep} 	% Space below
{\small}		% Body font
{}		% Indent amount
{\small\bfseries}% Theorem head font
{:}		% Punctuation after theorem head
{.5em}	% Space after theorem head
{}		% Theorem head spec (can be left empty, meaning ‘normal’)

\usepackage{fancyhdr}%% Cambia il carattere delle didascalie delle figure %%
\usepackage[font=small,format=plain,labelfont=bf,up,textfont=it,up]{caption}

\usepackage{comment}

%per le tabelle lunghe e particolari
\usepackage{lscape}

\makeatother

\usepackage{babel}
\begin{document}
\tableofcontents{}

\selectlanguage{english}%
\global\long\def\veraw#1#2#3{#1\models_{#2}#3}


\global\long\def\vera#1#2{#1\models#2}


\global\long\def\nonvera#1#2{#1\nvDash#2}


\global\long\def\nonveraw#1#2#3{#1\nvDash_{#2}#3}


\global\long\def\nonSem#1#2{#1\nvdash#2}


\global\long\def\nonTeor#1#2{\nvdash_{#1}#2}


\global\long\def\nonSemW#1#2#3{#1\nvdash_{#2}#3}


\global\long\def\verita#1#2{#1\in V(#2)}


\global\long\def\entail#1#2{#1\models#2}


\global\long\def\semantica#1#2#3{#1\vdash_{#2}#3}


\global\long\def\teolm#1#2{\vdash_{#1}#2}


\global\long\def\semGen#1#2{#1\vdash#2}


\global\long\def\boxx#1{\square#1}


\global\long\def\diam#1{\diamond#1}


\global\long\def\dia{\diamond a}


\global\long\def\boa{\boxx a}


\global\long\def\noa{\neg a}


\global\long\def\forhten#1#2#3{\forall#1#2\implies#3}


\global\long\def\implica#1#2{#1\implies#2}


\global\long\def\teorema#1{\vdash_{\Lambda}#1}


\global\long\def\teorGamma#1{\Gamma\vdash_{\Lambda}#1}


\global\long\def\teoa{\teorGamma a}


\global\long\def\teoremaDi#1{\vdash_{#1}}


\global\long\def\consist{\mbox{\ensuremath{\Lambda}}-consistente}


\global\long\def\consMax{\Lambda-consistente\: massimale}


\global\long\def\veraCanAlfa#1{\vera{M^{\Lambda}}{_{\alpha}}#1}


\global\long\def\veraCA{\vera{M^{\Lambda}}{_{\alpha}}a}


\global\long\def\veraCan#1#2{\vera{M^{\Lambda}}{_{#1}}#2}


\global\long\def\nonveraCan#1#2{\nonveraw{M^{\Lambda}}{#1}{#2}}


\global\long\def\consMaxLog#1{#1-consistente\ massimale}


\global\long\def\relazCAB#1#2{\{a\ |\ \boa\in#1\}\subseteq#2}


\global\long\def\relazCAD#1#2{\{\diamond b\ |\ b\in#2\}\subseteq#1}


\global\long\def\andoria#1{a_{1}\wedge...\wedge a#1}


\global\long\def\andbox#1{\boa_{1}\wedge\dots\wedge\boxx a#1}


\global\long\def\neci#1#2{[#1]#2}


\global\long\def\posi#1#2{<#1>#2}


\global\long\def\necf#1{[F]#1}


\global\long\def\posf#1{<F>#1}


\global\long\def\necp#1{[P]#1}


\global\long\def\posp#1{<P>#1}


\global\long\def\verins#1{\Vert#1\Vert^{\mu}}


\global\long\def\verinsx#1#2{\Vert#1\Vert^{\mu^{#2}}}


\global\long\def\cardinal#1{\Vert#1\Vert}


\global\long\def\alue{AL\mbox{\ensuremath{\mathcal{UE}}}}


\global\long\def\alc{AL\mbox{\ensuremath{\mathcal{\mathcal{C}}}}}
	

\global\long\def\globally#1{\mathcal{G}#1}


\global\long\def\eventually#1{\mathcal{F}#1}


\global\long\def\next#1{\mathcal{X}#1}


\global\long\def\until#1#2{#1\mathcal{U}#2}


\global\long\def\release#1#2{#1\mathcal{R}#2}
\selectlanguage{italian}%




\chapter{Introduzione}

\label{cap:introduction}


\section{Intro}

Se voi signorine finirete questo corso, e se sopravviverete sarete
dispensatori di fbf e pregherete per modellizzare sistemi assurdi
in modo ancora più assurdo, ma fino a quel giorno non siete altro
che buoni annulla convinti che tutti i cretesi sono stupidi e forse
mentono.

Lasciate il formaggio fuori dall'aula.


\global\long\def\veraw#1#2#3{#1\models_{#2}#3}
 

\global\long\def\vera#1#2{#1\models#2}


\global\long\def\verita#1#2{#1\in V(#2)}
 

\global\long\def\entail#1#2{#1\models#2}
 

\global\long\def\semantica#1#2#3{#1\vdash_{#2}#3}


\global\long\def\semGen#1#2{#1\vdash#2}


\global\long\def\boxx#1{\square#1}


\global\long\def\diam#1{\diamond#1}


\global\long\def\dia{\diamond a}


\global\long\def\boa{\boxx a}


\global\long\def\forhten#1#2#3{\forall#1#2\Rightarrow#3}


\global\long\def\implica#1#2{#1\Rightarrow#2}



\chapter{Introduction}

$a$ è vera nel mondo $\alpha$, e scriviamo $\mu\models_{\alpha}a$

se
\begin{itemize}
\item $a$ è una lettera enunciativa allora deve valere $\verita a{\alpha}$ 
\item $a$ è del tipo: $a\lor b$ .... allora.... $\mu\models_{\alpha}a$
oppure $\mu\models_{\alpha}b$
\end{itemize}

\section{Formule di Logica modale e significato}

\begin{tabular}{|c|c|c|}
\hline 
$\diam{}{a\Rightarrow}\boxx a$  & funzione parziale & $\forhten{\alpha}{:\,\alpha R\beta,\:\beta R\gamma}{\beta}=\gamma$\tabularnewline
\hline 
\end{tabular}

Funzione parziale, dimostrazione

.

Ip) funzione parziale 

Ts) $\diam{}{a\Rightarrow}\boxx a$

.

$\diam{}a$ falsa allora dato che l'antecedente è falso di ha $\implica{\diam{}a}{\boxx a}$

$\diam{}a$ vera allora $\exists\beta$:$\alpha R\beta$ e$\in V(\beta)$,
ma dato che la funzione è parziale questo $\beta$ è unico !

da cui $\vera{\mu}{\implica{\diamond a}{\boxx a}}$

.

.

Ip) $\diam{}{a\Rightarrow}\boxx a$

Ts) funzione parziale

.

.

Per assurdo: suppongo non che la funzione non sia parziale. Se è così
$\exists\alpha:$ $\alpha R\beta,$ $\alpha R\gamma$, considero un
modello in cui V(A) = \{$\beta$ \} , $\boxx A$ non vale in $\alpha$
dato che A è falsa in $\gamma$, il che contraddice l'ipotesi (BAM!)\\\\

\begin{tabular}{|c|c|c|}
\hline 
$\diam{}{a\iff}\boxx a$  & funzione totale & $\forall\alpha\exists\,!\,\beta:\:\alpha R\beta$ \tabularnewline
\hline 
\end{tabular}\\\\

non ci sono ``conti'' da fare, R è seriale sse R è seriale $\boxx a\implies\diam a$
, e se R è una funzione parziale $\implica{\diam a}{\boxx a}$

quindi dato che l'implica prevede un and di implica da una parte e
dall'altra per definizione abbiamo la tesi

.

.

\begin{tabular}{|c|c|c|}
\hline 
$\diam{}{a\Rightarrow}\boxx{\diam a}$  & relazione euclidea & $\forhten{\alpha,\beta,\gamma}{:\:(\alpha R\beta,\:\alpha R\gamma)}{\beta}R\gamma$
da cui anche: $\beta$R$\beta$, $\gamma R\gamma$, $\gamma$R$\beta$\tabularnewline
\hline 
\end{tabular} \\\\

Ip) relazione euclidea

Ts) $\diam{}{a\Rightarrow}\boxx{\diam a}$ 

Suppongo sia vero l'antecedente (se falso ho finito), quindi vale:
$\dia$ da cui: $\vera{\mu}{\dia}$

dato che $\dia$ si ha che esiste almeno un $\beta$ tale che in beta
vale a 

solo un beta: autoanello perché euclidea e quindi $\boxx{\dia}$

diversi beta: ognuno dei vari $\beta'$, $\beta''$ , ecc. sono in
relazione con $\beta$, dato che la relazione è euclidea, pertanto
dato che in $\beta$ vale $a$, in ognuno di loro vale $\dia$ \\

Ip)$\diam{}{a\Rightarrow}\boxx{\diam a}$ 

Ts) relazione euclidea

Per assurdo, suppondo valga ip) ma non la tesi

Considero un Frame in cui: $\alpha R\beta,$ $\alpha R\gamma,$ $\beta R\gamma$
ma NON $\beta R\gamma$ cioè si ha un frammento in cui non vale l'euclidea.
Poniamo che il modello sia tale che $V(A)$$=\{\gamma\}$

In queste ipotesi vale $\dia$ dato che in $\gamma$ vale $a$. In
$\beta$ non vale $a$ e neppure $\dia$ perché non ha ``uscite'',
da cui in $a$ non vale $\boxx{\dia}$ contraddicendo così l'ipotesi
(BAM!) \\\\


\section{Semantica}

$\semGen ab$ cioè a è conseguenza semantica di b, se in ogni Frame,
Modello e Mondo in cui $\vera{\mu}b$ si ha anche $\vera{\mu}a$\\

$\dia\equiv\sim\boxx{\sim a}$

Vale da sinistra a destra,

Infatti:

se $\veraw{\mu}{\alpha}{\dia}$ allora 

$\exists\beta:$$\alpha R\beta$ e $\veraw{\mu}{\beta}a$ da cui:

$\mu\nvDash_{\beta}\sim a$

per questo in $\alpha$ non vale $\boxx{\sim a}$ (perché non vale
$\sim a$ in $\beta$)

allora in $\alpha$ vale $\sim\boxx{\sim a}$ cioè $\veraw{\mu}{\alpha}{\sim\boxx{\sim a}}$
cioè la tesi. \\

Vale anche da destra a sinistra, dimostrazione simile.



\chapter{Semantica}


\section{Simboli secessari}

$\semGen ab$ cioè a è conseguenza semantica di b, se in ogni Frame,
Modello e Mondo in cui $\vera\mu b$ si ha anche $\vera\mu a$\\


$\dia\equiv\neg\boxx{\neg a}$

Vale da sinistra a destra,

Infatti:

se $\veraw\mu\alpha\dia$ allora

$\exists\beta:$$\alpha R\beta$ e $\veraw\mu\beta a$ da cui:

$\mu\nvDash_{\beta}\neg a$

per questo in $\alpha$ non vale $\boxx{\neg a}$ (perché non vale
$\neg a$ in $\beta$)

allora in $\alpha$ vale $\neg\boxx{\neg a}$ cioè $\veraw\mu\alpha{\neg\boxx{\neg a}}$
cioè la tesi. \\


Vale anche da destra a sinistra, dimostrazione simile. \\



\section{Logiche}

Una logica $\Lambda$ su L è un insieme di fbf su L che: 
\begin{itemize}
\item contiene tutte le tautologie 
\item è chiusa rispetto al Modus Ponens 
\end{itemize}
Ad esempio; $PL(\phi)$ cioè i teoremi della logica proposizionale

Altro esempio $\Lambda_{C}=\{a\,|\,\vera F{a\: per}\ ogni\ F\in C\}$

infatti: 
\begin{itemize}
\item contiene tutte le tautologie perché sono vere mondo per mondo dappertutto 
\item MP : suppongo che in un mondo $\alpha$ accada che: $\nonveraw{\mu}{\alpha}b$
, $\veraw{\mu}{\alpha}a$ . Se vale anche $\veraw{\mu}{\alpha}{\implica ab}$
... l'antecedente è vero, quindi dato che l'implicazione è vera, deve
essere vero anche il conseguente da cui non può che essere $\veraw{\mu}{\alpha}b$ 
\end{itemize}
Una logica si dice \textbf{uniforme }se è chiusa rispetto a sostituzioni
uniformi cioè se sostituendo a una lettere uguali formule uguali in
una tautologia, ottengo una tautologia.

Es. $\Lambda_{C}=\{a\,|\,\vera F{a\: per}\ ogni\ F\in C\}$ NON è
uniforme infatti se considero $V(A)=S$, dove S sono tutti gli stati
possibili (mondi), vale anche $\veraw{\mu}{\alpha}A$, e cioè A è
una tautologia, se al posto di A sostituisco $B\wedge\neg B$ (falsa
in ogni modello e mondo) non ottengo una tautologia.\\
 \\


\textbf{\emph{\large{{{{Teorema}}}}}}{\large \par}

Sono equivalenti: 
\begin{enumerate}
\item $\Lambda$ è normale 
\item per ogni intero n $\geq0$,


$\teorema{a1\wedge a2\wedge...\wedge an}\implies a$ implica $\teorema{\boa1\wedge\boa2\wedge...\wedge\boa n}\implies\boa$

\item valgono:

\begin{enumerate}
\item $\teorema{\boxx T}$ 
\item $\teorema{\boa\wedge\boxx b}\implies\boxx{(a\wedge b)}$ 
\item $\teorema{\implica ab}$ implica $\teorema{\boa\implies\boxx b}$ 
\end{enumerate}
\end{enumerate}
Dimostrazione

$1\implies$2

per induzione.

se n = 0 allora $\teorema a$ allora $\teorema{\boa}$ per la regola
RN che vale in $\Lambda$ per ipotesi

se n > 0 (passo induttivo) suppongo valga l'antecedente, altrimenti
2 vale senz'altro;

Ricordiamo che $a1\wedge a2\wedge...\wedge an\implies a\equiv a1\wedge a2\wedge...a_{n-1}\implies(an\implies a)$ 



\chapter{Verso la decidibilità - Logica determinata}


\section{Insieme $\Lambda$ consistente e sue proprietà}

Sia $\Lambda$ una logica (cioè ha tutte le tautologie ed è chiusa
rispetto al Modus Ponens)

$\Gamma$ si dice $\Lambda$-consistente se: $\nonSemW{\Gamma}{\Lambda}{\bot}$,
dove $\bot=A\wedge\neg A$

$\Delta$ si dice $\Lambda$-consistente massimale se per ogni fbf
$a$ $a\in\Delta$ oppure $\neg a\in\Delta$ $ $\\


\textbf{Proprietà:} $ $
\begin{enumerate}
\item Se $\teoa$ e $\Gamma\subseteq\Delta$ allora $\Delta\teorema a$.
Ovvero se alcune premesse non mi servono posso comunque metterle per
dedurre una formula 
\item Se $\teorGamma a$ e $\Lambda\subseteq\Lambda'$ allora $\Gamma\vdash_{\Lambda'}a$.
Ovvero quello che posso dedurre in una logica più scarna (es. PL)
lo posso dedurre anche in una più ricca che la contien (es. Modale) 
\item se $a\in\Gamma$ allora $\teoa$ . \\
 Infatti $\teorema{a\implies a}$ è un teorema dato che $a\implies a$
è una tautologia 
\item $\{a|\teoa\}$ è la minima logica che contiene $\Gamma\cup\Lambda$.
Infatti posso dedurre tutte le tautologie da $\Gamma$, anche se non
userò nessuna formula di $\Gamma$ ma solo quelle che già sono nella
logica $\Lambda$ $ $
\item Se $\teoa$ e $\{a\}$$\teorema b$ allora $\teorGamma b$ \\
 Infatti: per dedurre $a$ uso regole di inferenza, formule di $\Gamma$,
assiomi di $\Lambda$. Per arrivare in $b$ uso assiomi di $\Lambda$
e regole di inferenza, quindi posso arrivare da $\Gamma$ direttamente
in $b$ usando formule di $\Gamma$, regole di inf. e assiomi di $\Lambda$ 
\item Se $\teoa$ e $\teorGamma{\implica ab}$ allora $\teorGamma b$, dato
che $\Lambda$ è chiusa rispetto al MP 
\item $\Gamma\cup\{a\}\teorema b$ se e solo se $\teorGamma{\implica ab}$
\\
 \textbf{Andata}: $\teorema{a_{1}\wedge...\wedge a\wedge...\wedge}a_{n}\implies b$
(per definizione di teorema), si può portare $a$ alla destra dell'implicazione
$\teorema{a_{1}\wedge...\wedge}a_{n}\implies(a\implies b)$ \\
 \textbf{Ritorno}: $\teorema{a_{1}\wedge}...\wedge a_{n}\implies(a\implies b)$,
basta portare $a$ tra le $ $and. 
\item $\teoa$ se e solo se $\Gamma\cup\{\neg a\}$ non è $\Lambda$-consistente
\\
 \\
 \textbf{Andata}: $\teoa$, $\Gamma\teorema{\neg a}$, posso dedure
$\bot$ che è contro la definizione di $\Lambda$-consistenza\\
 \textbf{Ritorno}: Se$ $$\Gamma\cup\{\neg a\}$ non è $\Lambda$-consistente,
allora $\Gamma\cup\{\neg a\}\teorema{\bot}$ da cui per 7. \\
 $\Gamma\teorema{\neg a\implies\bot}$ (sposto $\neg a$ a destra
e metto l'implica), \\
 Dato che $(\neg a\implies\bot)\implies a$ è una tatutologica, per
MP ottengo\\
 $a$ 
\item $\Gamma$ è $ $$\consist$ se e solo se $\exists\beta:\nonSem{\Gamma}{_{\Lambda}\beta}$
\\
 \textbf{Andata}: Basta prendere $\neg a\wedge a$\\
 \textbf{Ritorno}: Se deducessi tutte le formule ($\neg$$\exists\beta:\nonSem{\Gamma}{_{\Lambda}\beta}$
significa $\forall\beta:\teorGamma{\beta}$) , potrei dedurre anche
$\bot$, da cui la non consistenza 
\item $\Gamma$ è $\consist$ se per ogni $a$ \\
 $\Gamma\cup\{a\}$ o $\Gamma\cup\{\neg a\}$ è $\consist$\\
 se $\teoa$ allora $ $$\Gamma\cup\{\neg a\}$ non è consistente
perché con $a$ e $\neg a$ posso dedurre $\bot$, ma $\Gamma\cup\{a\}$
lo è \\
 se $\Gamma\teorema{\neg a}$ allora $ $$\Gamma\cup\{\neg a\}$ è
consistente ma non $\Gamma\cup\{a\}$ 
\item $\bot$$\notin\Gamma$ se $\Gamma$ è $\consist$ (altrimenti potrei
dedurlo per il 3.) 
\item Se $\Delta$è $\consist\: massimale$ e $\Delta\teorema a$ allora
$a\in\Delta$\\
 se $a\notin\Delta$ allora $\neg a\in\Delta$ (dato che $\Delta$è
massimale) \\
 ma se $\Delta$ contiene $\neg a$ allora per il 2.)\\
 $\Delta\teorema{\neg a}$ , che insieme a $\Delta\teorema a$ mi
da $\Delta\teorema{\bot}$ 
\item Se $\Delta$ è $\consMax$ e $ $$a\in\Delta$. $\implica ab\in\Delta$
allora $b\in\Delta$. \\
 Lo si vede subito usando 2.) se tutti e tre, e poi 6.) (deduco $a$,
$\implica ab$, allora deduco anche $b$) 
\end{enumerate}

\section{Insieme $\Lambda$ consistente massimale}

\emph{\large{{{Lemma di Lindelman - Esistenza dell'insieme $\consMax$}}}}{\large{{
}}}\emph{\large{{{in una logica $\Lambda$}}}}{\large{{ }}}\emph{\large{{{consistente}}}}{\large{{}}}\\
 {\large{{ }}}\\
 {\large{{ Considero tutte le formule $b1,\ b2,\ b3,\dots$ della
logica $\Lambda$ (posso farlo perché sono un'infinità numerabile)}}}{\large \par}

Chiamo $\Gamma_{0}$ un insieme che contiene una sola formula (ad
esempio una tautologia)

Dopodichè iterativamente, per ogni formula mi chiedo\\


$\Gamma_{0}$$\teorema{b1}$ ? $\begin{cases}
si: & \Gamma_{1}=\Gamma_{0}\cup b1\\
no: & \Gamma_{1}=\Gamma_{0}\cup\neg b1
\end{cases}$\\


$\Gamma_{1}$$\teorema{b2}$ ? $\begin{cases}
si: & \Gamma_{2}=\Gamma_{1}\cup b2\\
no: & \Gamma_{2}=\Gamma_{1}\cup\neg b2
\end{cases}$ 
\begin{description}
\item [{$\Delta=\bigcup_{n\geq0}\Gamma_{i}$}] (nota, questa unione è infinita) 
\item [{$\Delta$}] è consistente massimale infatti:\end{description}
\begin{enumerate}
\item Massimale in quanto contiene $a$ oppure $\neg a$ per costruzione 
\item Consistente. Per assurdo se non lo fosse avrei: $\Delta\teorema{\bot}$\\
 cioè esiste un numero finito di formule di $\Delta$ da cui deduco
il falso,\\
 dato che è un numero finito di formule, sta in $\Gamma_{i}$ , cioè
esiste un $\Gamma_{i}$ non consistente, assurdo perché lo sono tutti
per costruzione \lightning 
\end{enumerate}
\emph{\large{{Nota:}}}{\large \par}

 
\begin{itemize}
\item Non sappiamo costruire $\Delta$ perché nasce da unione infinita 
\item Non è unico, infatti se considero formule in ordine diverse potrei
``dire'' si o no in modo diverso \\
 es. $a,\ \implica ab,\ b$ (allora $\Delta$ contiene $b$)\\
 es. $b,\ c$ (allora $\Delta$ contiene $\neg b$) 
\end{itemize}

\subsection{Teorema}

$\teoa$ se e solo se $a\in$ a tutti i quei $\Delta$ $\Lambda-consistenti\ massimali$
tali che: $\Gamma\subseteq\Delta$\\


\textbf{Andata:}

$\teoa$, anche $\Delta\teorema a$ per la 1.)

\textbf{Ritorno:}

Per assurdo, se $\nonSem{\Gamma}{_{\Lambda}a}$ allora $\Gamma\cup\{\neg a\}$
è $\consist$ (per la 8.)

da cui per Lindellman esiste $\Delta'$ che contiene $\Gamma\cup\{\neg a\}$
consistente massimale

data la consistenza $\Delta'$ non contiene $a$, il che è contro
l'ipotesi \lightning
 
\end{document}
