%% STILE DELLE INTESTAZIONI %%


\pagestyle{fancy} \global\long\def\chaptermark#1{\markboth{#1}{}}
 % aggiungi \thechapter.\ per anche il numero capitolo
%\renewcommand{\sectionmark}[1]{\markright{#1}} % titoli di sezione
\fancyhf{} \fancyhead[RE,RO]{\small\thepage} \fancyhead[LE,LO]{\small\em\leftmark}
% \leftmark = 2.TitoloCapitolo, \rightmark = TitoloSezione
%\fancyhead[LO]{\small\em\rightmark}  % LO = Left Odd, RO = Right Odd, LE = Left Even, RE = Right Even
\fancypagestyle{plain}{ % titolo di sezione e simili
 \fancyhf{} % remove everything
 \global\long\def\headrulewidth{0pt}
 % remove lines as well
 \global\long\def\footrulewidth{0pt}
 }

%%% Cambia il carattere dei titoli di sezione %%
%\usepackage{titlesec}
%\titleformat{\section}{\Large\bfseries\sffamily}{\thesection}{1em}{}


%% Stile dei titoli di capitolo %%
\makeatletter \global\long\def\thickhrulefill{\leavevmode\leaders\hrule height 0.7ex \hfill\kern \z@}
 \global\long\def\@#1#2#3#4#5#6#7#8#9{%
\vspace*{10\p@}%
{\parindent\z@ \centering\reset@font %thickhrulefillparnobreakvspace{3p@}
{\huge\bfseries\rmfamily\strut\thechapter.\ #1}\par\nobreak\interlinepenalty\@M \hrule\vspace*{10\p@}%
\vskip30\p@ }}


\global\long\def\@#1#2#3#4#5#6#7#8#9{%
\vspace*{10\p@}%
\vspace*{10\p@}%
{\parindent\z@ \centering\reset@font %thickhrulefillparnobreakvspace{3p@}
{\huge\bfseries\rmfamily\strut#1}\par\nobreak\interlinepenalty\@M \hrule\vspace*{10\p@}%
\vskip30\p@ }}

