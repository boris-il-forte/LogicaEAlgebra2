\begin{savequote}[50mm]
---Bu-burro, certo, ci vuole un po' di burro! Buuuurro!--- \\
---Bu-bu-burro? Eccolo!--- \\
---Oh, grazie! Burro! Benissimo!---
\qauthor{Lewis Carroll} \end{savequote}


\chapter{Logica Modale Del Prim'Ordine}


\section{La Sintassi delle Logiche Modali del Prim'Ordine}

Una logica modale del prim'ordine è definita sul seguente alfabeto:
\begin{itemize}
\item Costanti $(k_{1},...,k_{n})$
\item Variabili $(x_{1},...,x_{n})$
\item Lettere predicative $A_{i}^{j}$
\item Connettivi logici: $\neg,\,\wedge,\,\vee,\,\implies,\,\iff$
\item Quantificatori universali ed esistenziali: $\forall x_{i},\,\exists x_{i}$
\item Connettivi modali $\square,\,\diamond$
\item Parentesi ), (
\end{itemize}
Si chiamano termini le costanti e le lettere predicative, gli indichiamo
come $t_{i}$

Si chiamano formule atomiche, le formule del tipo:

$A_{i}^{n}(t_{1},...,\, t_{n})$

Le formule ben formate sono definite come al solito:
\begin{itemize}
\item Le formule atomiche sono formule ben formate
\item $a,\,\neg a,\,(\forall x)a,\,(\exists x)a,\,\boa,\,\dia$ sono furmule
ben formate
\item $a,\, b,\,(a\wedge b),\,(a\vee b),\,(a\implies b),\,(a\iff b)$ sono
formule ben formate
\item Null'altro è una formula ben formata.
\end{itemize}

\section{Semantica della Logica Modale del Prim'Ordine}


\subsection{Frame e modello}

Definiamo un frame della logica modale del prim'ordine il seguente:

F = (S, R, D)

in cui:
\begin{itemize}
\item S è insieme dei possibili stati
\item R è la relazione di raggiungibilità tra gli stati
\item D è la funzione che associa a ogni stato il suo dominio.
\end{itemize}
Chiamiamo modello il seguente:

$\mu=(S,\, R,\, D,\, I)$

dove I è la funzione di interpretazione, che può essere vista come
l'unione di due funzioni separate:
\begin{itemize}
\item $I_{c}$ è la funzione di interpretazione delle costanti, ossia che
in ogni stato associa al valore di una costante un valore del dominio
in quello stato
\item $I_{P}$ è la funzione di interpretazione dei predicati, che associa
a un predicato una relazione n-aria tra gli elementi del dominio
\end{itemize}

\subsection{Tipi di dominio}

I domini nella logica modale del pr'imordine possono essere visti
in verie maniere differenti, a seconda che gli stati abbiano domini
uguali o diversi tra loro.

In generale, infatti ogni stato potrebbe avere domini a piacere differenti.

Si dicono domini costanti quei domini per cui vale:

$\forall\alpha,\beta\in S\, D_{\alpha}=D_{\beta}$

Si dicono domini variabili monotoni i domini per cui vale:

$\forall\alpha,\beta\in S\, D_{\alpha}\subseteq D_{\beta}$

Si dicono domini variabili antimonotoni i domini per cui vale:

$\forall\alpha,\beta\in S\, D_{\alpha}\supseteq D_{\beta}$

I domini monotoni sono i domini in cui ``nulla si crea'', mentre
i domini antimonotoni sono quelli in cui ``nulla si distrugge''.
Un dominio costante è sia monotono che antimonotono, per cui gli elementi
dell'insieme rimangono immutabili.

Se chiamiamo P l'insieme dei predicati, C l'insieme delle costanti,
e $R_{D}$ l'insieme di tutte le possibili relazioni sui domini D,
possiamo definire l'interpretazione I come:

$I_{c}:\, C\times S\longrightarrow D$

$I_{P}:\, P\times S\longrightarrow R_{D}$

In questo caso si parla di logica a designatori non rigidi.

Se invece consideriamo una logica a designatori non rigidi abbiamo
I definita come:

$I_{c}:\, C\longrightarrow D$

$I_{P}:\, P\longrightarrow R_{D}$


\subsection{Semantica}

Chiamiamo s la funzione di assegnamento, posto V l'insieme delle variabili
ossia la funzione, è la seguente:

$s:\, V\times S\longrightarrow D$

è possibile estendere s ad s{*}, e se chiamiamo T l'insieme di tutti
i possibili termini, abbiamo che:

$s*:\, T\times S\longrightarrow D$

che è così definita:

$s*(x_{i})=s(x_{i})$

$s*(k_{i})=I(k_{i})$

Indichiamo che una formula è vera in un modello $\mu$, rispetto all'assegnamento
s, in un mondo $\alpha$ nel seguente modo:

$\veraw{\mu,s}{\alpha}a$

La verità di una formula è definita per induzione sui connettivi minimi:
\begin{itemize}
\item $\veraw{\mu,s}{\alpha}{A_{i}^{n}(t_{1},...,t_{n})\iff(s*(t_{1}),...,s*(t_{n}))\in I_{P}(A_{i}^{n},\alpha)}$
\item $\veraw{\mu,s}{\alpha}{\neg b}\iff\nonveraw{\mu,s}{\alpha}b$
\item $\veraw{\mu,s}{\alpha}{b\implies c}\iff\veraw{\mu,s}{\alpha}{c\,\vee\,\nonveraw{\mu,s}{\alpha}b}$
\item $\veraw{\mu,s}{\alpha}{(\forall x)b}\iff(\forall s'\,:\, s'(x)\not=s(x)\,\wedge\, s'(x_{i})=s(x_{i}))\,\veraw{\mu,s}{\alpha}{(\forall x)b}$
\item $\veraw{\mu,s}{\alpha}{\boxx b}\iff\forall\beta\,:\,(\alpha,\,\beta)\in R\,\veraw{\mu,s}{\beta}b$
\end{itemize}
Una formula M è vera in un mondo se ogni assegnamento la soddisfa,
insoddisfacibile se nessuno la sofddisfa e sodisfacibile se è vera
per alcuni assegnamenti.

Una formula si dice vera in un modello se è vera in tutti i mondi
del modello

Una formula si dice valida in un frame se è vera in tutti i modelli
costruiti su quel frame.


\section{Assiomatizzazzione della Logica Modale del Prim'Ordine}


\subsection{Gli Assiomi della logica del prim'ordine}

Gli assiomi della logica del prim'ordine sono:
\begin{itemize}
\item A1: $a\implies(b\implies a)$
\item A2: $(a\implies(b\implies c))\implies((a\implies b)\implies(a\implies c))$
\item A3: $(\neg a\implies\neg b)\implies((\neg a\implies b)\implies a)$
\item A4: $(\forall x)a(x)\implies a[t/x]$, dove t è un termine libero
per x in A(x)
\item A5: $(\forall x)(a\implies b)\implies(a\implies(\forall x)b)$, purchè
non ci siano occorrenze libere di x in A
\end{itemize}
Inoltre valgono le due regole di inferenza:
\begin{itemize}
\item MP: $\dfrac{a,\, a\implies b}{b}$
\item Gen: $\dfrac{a}{(\forall x)a}$
\end{itemize}

\subsection{Formula di Barcan}

Si chiama formula di barcan la seguente formula:

$(\forall x)\boa\implies\boxx{(\forall x)a}$

o equivalentemente la sua duale:

$\diam{(\exists x)a\implies(\exists x)\dia}$

Si chiama formula di barcan inversa la formula:

$\boxx{(\forall x)a}\implies(\forall x)\boa$

o equivalentemente la sua duale:

$(\exists x)\dia\implies\diam{(\exists x)a}$

La formula di barcan vale se e solo se il modello è antimonotono,
la formula di barcan inversa vale se e solo se il modello è monotono.

Quindi in un modello a designatori costanti vale la formula di barcan
inversa.


\subsection{Minima logica modale del prim'ordine}

Se consideriamo la logica che usa tutti gli assiomi della logica modale
e quella del prim'ordine, ossia una logica che usa i seguenti assiomi:
\begin{itemize}
\item A1, A2, A3
\item A4, A5
\item K
\end{itemize}
E le seguenti regole di inferenza:
\begin{itemize}
\item MP
\item Gen
\item RN
\end{itemize}
Otteniamo la minima logica modale del prim'ordine.

Si può dimostrare che in questa logica vale la formula di Barcan inversa:

$\teorema{(\forall x)a\implies a}$ -- per A4, t = x

$\teorema{\boxx{((\forall x)a\implies a)}}$ -- per RN

$\teorema{\boxx{((\forall x)a\implies a)\implies(\boxx{(\forall x)a}\implies\boa)}}$
-- per K

$\teorema{\boxx{(\forall x)a}\implies\boa}$ -- per MP

$\teorema{(\forall x)(\boxx{(\forall x)a}\implies\boa)}$ -- per Gen

$\teorema{(\forall x)(\boxx{(\forall x)a}\implies\boa)}\implies(\boxx{(\forall x)a}\implies(\forall x)\boa)$
-- per A5 

$\teorema{\boxx{(\forall x)a}\implies(\forall x)\boa}$ -- per MP

che è la formula di barcan inversa.

Allo stesso modo si può dimostrare che aggiungendo lo schema:

B: $a\implies\boxx{\dia}$

Dalla minima logica modale del prim'ordine si ricava anche la formula
di Barcan. Questo è vero intuitivamente perchè se ho una relazione
simmetrica monotona, allora deve anche essere antimonotona, e questo
può essere vero solo se ci troviamo nel caso a domini costanti.

Si può usare, anzichè la logica normale, una logica classica, e si
possono ricavare logiche in cui le due formule di Barcan non valgano.
