\begin{savequote}[60mm]
Se io avessi un mondo come piace a me, là tutto sarebbe assurdo: niente sarebbe com'è, perché tutto sarebbe come non è, e viceversa! Ciò che è, non sarebbe e ciò che non è, sarebbe! 
\qauthor{Lewis Carroll} \end{savequote}


\chapter{Logiche modali particolari e Determinatezza}

Per il teorema di correttezza e completezza abbiamo che $\teorema a$
se e solo se $\vera{M^{\Lambda}}a$

Il problema di $\vera{M^{\Lambda}}a$ è che non so costruire a livello
``pratico'' il modello canonico dato che i suoi mondi sono infiniti.

Provo quindi a vedere se $\teorema a$ se e solo se $\vera{F^{\Lambda}}a$
possa valere almeno per particolari classi di frame.\\


\shadowbox{\parbox[c]{1\textwidth}{%
\textbf{Nota: }Per dimostrare la determinatezza di una logica rispetto
a una classe di frame con una proprietà si può mostrare che la relazione
del Frame canonico costruito da quella logica gode della stessa proprietà.

Con questa chiave di lettura diamo alcune dimostrazioni di determinatezza.%
}}


\section{Serialità del frame canonico di KD}

$R^{KD}$ è seriale se: $\forall\alpha\in S^{KD}\exists\beta:(\alpha,\beta)\in R^{KD}$

TS) $\teoremaDi{KD}a$ se e solo se $R^{KD}$ è seriale

$\relazCAB{\alpha}{\beta}$ se e solo se $\alpha$ e $\beta$ sono
in relazione.

Vogliamo quindi provare che per ogni insieme $\{a\ |\ \boa\in\alpha\}$
esiste un $\beta$ che sia $KD-consistente\ massimale$ che lo contiene,

per farlo mi basta mostrare che esiste un insieme $\beta_{0}$ consistente
che lo contiene, poi per il teorema di Lindelmann saprò anche che
ne esiste uno consistente massimale.\\


$\beta_{0}=\{a\ |\ \boa\in\alpha\}$ , dimostro che $\beta_{0}$ è
consistente.

Se per assurdo non lo fosse

$\teoremaDi{KD}\andoria{_{n}\implies\bot}$, dove $a_{1},a_{2},...,a_{n}$
sono formule di $\beta_{0}$

(da cui spostando $a_{n}$ dopo l'implica)

$\teoremaDi{KD}\andoria{_{n-1}\implies\neg a_{n}}$ (uso la definizione
2. di logica normale)

$\teoremaDi{KD}\andbox{_{n}}\implies\boxx{\neg a_{n}}$

$\andbox{_{n}}\implies\boxx{\neg a_{n}}\in\alpha$ (dato che $\alpha$
è $\consMaxLog{KD}$ contiene tutti i teoremi di $KD$)

$\andbox{_{n}}\in\alpha$ per costruzione di $\beta_{0}$ (in $\beta_{0}$
valgono tutto le $\boxx x$ se $x$ vale in $\alpha)$

$\boxx{\neg a_{n}}\in\alpha$ (per MP dalle due precedenti)

$\neg\dia_{n}\in\alpha$

$\boa_{n}\in\alpha$ \textbf{({*})}

$\boa_{n}\implies\dia_{n}$ (schema seriale) \textbf{({*}{*})}

Per MP fra \textbf{({*}) }e\textbf{ ({*}{*}) }si ha $\dia_{n}\in\alpha$

\noindent \begin{flushleft}
Da cui $\alpha$ non consistente, assurdo. \lightning
\par\end{flushleft}


\section{\noindent Debole densità del frame canonico di K$X\diamond$}

\noindent \begin{flushleft}
$R$ è debolmente densa se: $\forhten{\alpha,\beta}{:\:(\alpha R\beta)}{\exists\gamma:\,(\alpha R\gamma\wedge\gamma R\beta)}$
\par\end{flushleft}

Chiamiamo K$X\diamond$ una logica costruita a partire dalla logica
$K$ aggiungendo lo schema $X\diamond$: $\dia\implies\diam{\diam a}$

La partenza del ragionamento è simile a quello del precedente in cui
come insieme considero:

$\gamma_{0}=\{a\ |\ a\in\alpha\}\cup\{\diamond b\ |\ b\in\beta\}$\\


Per assurdo:

$\teoremaDi{KX\diamond}\andoria{_{n}\wedge\diam{b_{1}\wedge...\wedge\diam{b_{n}}}}\implies\bot$

dove $a_{1},...,a_{n}$ sono formule di $\alpha$, e $b_{1},...,b_{n}$
sono formule di $\beta$

$\teoremaDi{KX\diamond}\andoria{_{n}}\implies\neg(\diamond b_{1}\wedge...\wedge\diamond b_{n})$

pongo $b=b_{1}\wedge...\wedge b_{n}$\\


\textbf{Teorema al volo:}

$\teoremaDi{KX\diamond}\neg(\diamond c\wedge\diamond d)\implies\neg\diamond(c\wedge d)$,
cioè:

\textbf{$\teoremaDi{KX\diamond}\diamond(c\wedge d)\implies\diamond c\wedge\diamond d$}\\
 infatti, sono tautologie della logica proposizionale: $\neg c\implies\neg c\vee\neg d$

e $\neg d\implies\neg c\vee\neg d$,

queste sono anche teoremi di $\Lambda$

dato che $\Lambda$ è normale si ha \foreignlanguage{english}{$\boxx{\neg c}\implies\boxx{\neg c\vee\neg d}$}
(definizione 3.3 di logica normale)

e lo stesso vale per la seconda $\boxx{\neg d}\implies\boxx{\neg c\vee\neg d}$

dato che il conseguente si ha per due antecedenti diversi allora:

\selectlanguage{english}%
$\boxx{\neg c}\vee\boxx{\neg d\implies}\boxx{\neg c\vee\neg d}$\foreignlanguage{italian}{
da cui negando e scambiando antecedente e conseguente:}

\selectlanguage{italian}%
$\neg(\boxx{\neg c\vee\neg d})\implies\neg(\boxx{\neg c}\vee\boxx{\neg d)}$,
sviluppo il $\neg$

$\diamond(c\wedge d)\implies\diamond c\wedge\diamond d$\\


Uso il teorema ottenendo (in un certo senso ``portiamo fuori'' il
$\diamond$)

\selectlanguage{english}%
$\teoremaDi{KX\diamond}\andoria{_{n}}\implies\neg\diamond(b_{1}\wedge...\wedge\diamond b_{n})$\foreignlanguage{italian}{
cioè (per come ho posto b)}

\selectlanguage{italian}%
$\teoremaDi{KX\diamond}\andoria{_{n}}\implies\neg\diamond b$

Uso la definizione 2. di logica normale e riscrivo:

$\teoremaDi{KX\diamond}\andbox{_{n}}\implies\boxx{\neg\diamond b}$

$\teoremaDi{KX\diamond}\andbox{_{n}}\implies\neg\diamond\diamond b$

dato che $\andbox{_{n}}\implies\neg\diamond\diamond b\in\alpha$

e che: $\andbox{_{n}\in\alpha}$

anche $\neg\diamond\diamond b\in\alpha$ (MP dalle due precedenti)

dato che $\diamond b\in\alpha$ anche $\diamond\diamond b\in\alpha$
per lo schema delle relazioni debolmente dense

il che ci porterebbe alla non massimalità di $\alpha$ contro l'ipotesi.
\lightning


\section{Riflessività del frame canonico di KT}

$R^{KT}$ è riflessiva se: $\forall\alpha,\,\alpha R^{KT}\alpha$

Vogliamo dimostrare che l'insieme $\{a\,|\,\boa\in\alpha\}$ di formule
è contenuto in $\alpha$

Se $\boxx a\in\alpha$ cioè se $\teoremaDi{KT}\boxx a$ allora

$\teoremaDi{KT}x$ dato che $\boxx{a\implies a}$ è uno schema della
logica $KT$ e quindi:

$a\in\alpha$.

Per questi motivi $\{a\,|\,\boa\in\alpha\}$ è in effetti contenuto
in $\alpha$

Da cui segue la tesi.


\section{Simmetria del frame canonico di KB}

$R^{KB}$è simmetrica se $\forhten{\alpha,\beta}{\,\alpha R^{KB}\beta}{\beta R^{KB}\alpha}$

Se $\beta R^{K4}\alpha$ allora:

$\{\diam b\,|\, b\in\alpha\}\subseteq\beta$

per definizione di R del modello canonico.

Dal momento che vale l'assioma B: $a\implies\boxx{\dia}$,

per ogni formula $b\in\alpha$ si ha che:

$\boxx{\diam b\in\alpha}$,

D'altra parte ogni volta che $b\in\alpha$, $\diamond b\in\beta$
dato che $\beta R^{K4}\alpha$ e quindi:

$\relazCAB{\alpha}{\beta}$ cioè $\alpha R\beta$


\section{Correttezza e completezza di KD}

La logica KD è corretta e completa rispetto alla classe dei Frame
seriali

Dimostrazione

Ip) $\vera Fa$ con F seriale

Ts) $\teoremaDi{KD}a$\\


Se a è un teorema di KD, è ricavato da una serie di formule che possono
essere applicazioni degli schemi A1, A2, A3, K, D oppure applicazioni
del modus ponens o della regola di necessitazione.

Supponiamo per assurdo che:

$\nonTeor{KD}a$

allora avremo che:

$\nonvera{\mu^{KD}}a$

e quindi:

$\nonvera{F^{KD}}a$

Poiché a non è vera nel frame canonico, non può essere vera in alcun
frame seriale, ma questo va contro l'ipotesi, assurdo. Allora la tesi
deve essere valida.
