\global\long\def\veraw#1#2#3{#1\models_{#2}#3}
 

\global\long\def\vera#1#2{#1\models#2}


\global\long\def\verita#1#2{#1\in V(#2)}
 

\global\long\def\entail#1#2{#1\models#2}
 

\global\long\def\semantica#1#2#3{#1\vdash_{#2}#3}


\global\long\def\semGen#1#2{#1\vdash#2}


\global\long\def\boxx#1{\square#1}


\global\long\def\diam#1{\diamond#1}


\global\long\def\dia{\diamond a}


\global\long\def\boa{\boxx a}


\global\long\def\forhten#1#2#3{\forall#1#2\Rightarrow#3}


\global\long\def\implica#1#2{#1\Rightarrow#2}



\chapter{Introduction}

$a$ è vera nel mondo $\alpha$, e scriviamo $\mu\models_{\alpha}a$

se
\begin{itemize}
\item $a$ è una lettera enunciativa allora deve valere $\verita a{\alpha}$ 
\item $a$ è del tipo: $a\lor b$ .... allora.... $\mu\models_{\alpha}a$
oppure $\mu\models_{\alpha}b$
\end{itemize}

\section{Formule di Logica modale e significato}

\begin{tabular}{|c|c|c|}
\hline 
$\diam{}{a\Rightarrow}\boxx a$  & funzione parziale & $\forhten{\alpha}{:\,\alpha R\beta,\:\beta R\gamma}{\beta}=\gamma$\tabularnewline
\hline 
\end{tabular}

Funzione parziale, dimostrazione

.

Ip) funzione parziale 

Ts) $\diam{}{a\Rightarrow}\boxx a$

.

$\diam{}a$ falsa allora dato che l'antecedente è falso di ha $\implica{\diam{}a}{\boxx a}$

$\diam{}a$ vera allora $\exists\beta$:$\alpha R\beta$ e$\in V(\beta)$,
ma dato che la funzione è parziale questo $\beta$ è unico !

da cui $\vera{\mu}{\implica{\diamond a}{\boxx a}}$

.

.

Ip) $\diam{}{a\Rightarrow}\boxx a$

Ts) funzione parziale

.

.

Per assurdo: suppongo non che la funzione non sia parziale. Se è così
$\exists\alpha:$ $\alpha R\beta,$ $\alpha R\gamma$, considero un
modello in cui V(A) = \{$\beta$ \} , $\boxx A$ non vale in $\alpha$
dato che A è falsa in $\gamma$, il che contraddice l'ipotesi (BAM!)\\\\

\begin{tabular}{|c|c|c|}
\hline 
$\diam{}{a\iff}\boxx a$  & funzione totale & $\forall\alpha\exists\,!\,\beta:\:\alpha R\beta$ \tabularnewline
\hline 
\end{tabular}\\\\

non ci sono ``conti'' da fare, R è seriale sse R è seriale $\boxx a\implies\diam a$
, e se R è una funzione parziale $\implica{\diam a}{\boxx a}$

quindi dato che l'implica prevede un and di implica da una parte e
dall'altra per definizione abbiamo la tesi

.

.

\begin{tabular}{|c|c|c|}
\hline 
$\diam{}{a\Rightarrow}\boxx{\diam a}$  & relazione euclidea & $\forhten{\alpha,\beta,\gamma}{:\:(\alpha R\beta,\:\alpha R\gamma)}{\beta}R\gamma$
da cui anche: $\beta$R$\beta$, $\gamma R\gamma$, $\gamma$R$\beta$\tabularnewline
\hline 
\end{tabular} \\\\

Ip) relazione euclidea

Ts) $\diam{}{a\Rightarrow}\boxx{\diam a}$ 

Suppongo sia vero l'antecedente (se falso ho finito), quindi vale:
$\dia$ da cui: $\vera{\mu}{\dia}$

dato che $\dia$ si ha che esiste almeno un $\beta$ tale che in beta
vale a 

solo un beta: autoanello perché euclidea e quindi $\boxx{\dia}$

diversi beta: ognuno dei vari $\beta'$, $\beta''$ , ecc. sono in
relazione con $\beta$, dato che la relazione è euclidea, pertanto
dato che in $\beta$ vale $a$, in ognuno di loro vale $\dia$ \\

Ip)$\diam{}{a\Rightarrow}\boxx{\diam a}$ 

Ts) relazione euclidea

Per assurdo, suppondo valga ip) ma non la tesi

Considero un Frame in cui: $\alpha R\beta,$ $\alpha R\gamma,$ $\beta R\gamma$
ma NON $\beta R\gamma$ cioè si ha un frammento in cui non vale l'euclidea.
Poniamo che il modello sia tale che $V(A)$$=\{\gamma\}$

In queste ipotesi vale $\dia$ dato che in $\gamma$ vale $a$. In
$\beta$ non vale $a$ e neppure $\dia$ perché non ha ``uscite'',
da cui in $a$ non vale $\boxx{\dia}$ contraddicendo così l'ipotesi
(BAM!) \\\\


\section{Semantica}

$\semGen ab$ cioè a è conseguenza semantica di b, se in ogni Frame,
Modello e Mondo in cui $\vera{\mu}b$ si ha anche $\vera{\mu}a$\\

$\dia\equiv\sim\boxx{\sim a}$

Vale da sinistra a destra,

Infatti:

se $\veraw{\mu}{\alpha}{\dia}$ allora 

$\exists\beta:$$\alpha R\beta$ e $\veraw{\mu}{\beta}a$ da cui:

$\mu\nvDash_{\beta}\sim a$

per questo in $\alpha$ non vale $\boxx{\sim a}$ (perché non vale
$\sim a$ in $\beta$)

allora in $\alpha$ vale $\sim\boxx{\sim a}$ cioè $\veraw{\mu}{\alpha}{\sim\boxx{\sim a}}$
cioè la tesi. \\

Vale anche da destra a sinistra, dimostrazione simile.
