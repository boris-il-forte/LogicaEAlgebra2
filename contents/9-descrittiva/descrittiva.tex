
\chapter{Logiche Descrittive	}


\section{Introduzione - Logica AL}

Le DL sono una famiglia di logiche per la rappresentazione della conoscenza
che possono essere utilizzate per rappresentare conoscenza terminologica,
dando ad essa una semantica formale ben definita.

Un sistema di rappresentazione della conoscenza (KR) fornisce i mezzi
per definire, gestire, manipolare e ragionare su basi di conoscenza
(KB).

In una KB ci sono TBox e ABox

I TBox descrivono la terminologia della KB (concetti, ruoli atomici
e concetti composti), gli ABox sono asserzioni su individui della
KB

TBox es. $Madre\equiv Donna\sqcap Genitore$

ABox es. $Donna(Paola)$\\


La famiglia di logiche descrittive più utilizzata è la famiglia AL
(Attributive Language).

Nella logica AL (logica descrittiva tra le più semplici) ci sono:
\begin{description}
\item [{A,B,C}] concetti
\item [{R}] ruoli
\item [{		$\neg$}] negazione (solo di concetti atomici)
\item [{$\sqcap$}] di concetti
\item [{\textmd{I}}] concetti sono:\end{description}
\begin{itemize}
\item Concetti atomici A,B
\item $\top$, $\bot$
\item negazione di concetti atomici: $\neg A$
\item and di concetti: $C\sqcap D$
\item quantificazioni: $\forall R.C$, $\exists R.\top$
\end{itemize}
Un'interpretazione I di una base di conoscenza\'{ }e una coppia $I=<\Delta^{i},\cdot^{I}>$
composta da un dominio di interpretazione $\Delta$I, detto dominio
di I e da una funzione di interpretazione $\cdot^{I}$ che associa:

ad ogni \textbf{nome} di individuo un elemento: 

$a^{I}\in\Delta^{I}$

ad ogni \textbf{concetto} C un sottoinsieme di $\Delta^{I}$

$I:\ C\rightarrow C^{I}\subseteq\Delta^{I}$

e ad ogni \textbf{ruolo} un sottoinsieme di $\Delta$I \texttimes{}
$\Delta$I :

$R:\ R^{I}\subseteq\Delta^{I}\times\Delta^{I}$

I rouli sono quindi relazioni binarie, una volta interpretati

es. $(Peter,\ Chris)^{I}\in HaFiglio^{I}$

		

Inoltre:

$\bot^{I}=\emptyset$, $\top^{I}=\Delta^{I}$

$(C\sqcap D)^{I}=C^{I}\cap D^{I}$

$(\forall R.C)^{I}=\{a^{I}\in\Delta^{I}:\ (a^{I},b^{I})\in R^{I}\implies b^{I}\in C^{I}$\} 

es. $\forall Possiede.Cosa$ cioè se a possiede un oggetto del dominio
quello deve essere una cosa (slavery is bad)

$(\exists R.C)^{I}=\{a^{I}\in\Delta^{I}:\ \exists b\in\Delta^{I}:\ (a^{I},b^{I})\in R^{I}\wedge b^{I}\in C^{I}\}$\\


es. $\exists HaFiglio.Femmina$ cioè l'insieme dei genitori con una
figlia femmina (grow up well kid)\\



\subsection{Varianti di AL}

Indebolendo la logica AL si ottengono le logiche poco descrittive
della famiglia FL:
\begin{itemize}
\item F L- è ottenuta da AL eliminando la negazione atomica •
\item FL0 è ottenuta da FL- eliminando anche la quantificazione esistenziale
\end{itemize}
La logica AL può essere estesa aggiungendo alcuni costruttori: 
\begin{itemize}
\item costrutto $\mathcal{U}$ disgiunzione dei concetti $C\sqcup D$
\item costrutto $\mathcal{E}$ quantificazione esistenziale qualificata
$\exists R.C$
\item costrutto $\mathcal{C}$ complemento di concetti complessi $\neg C$
\item costrutto $\mathcal{N}$ cardinalità di un ruolo 
\end{itemize}
$(\leq nR)^{I}=\{a^{I}:\ \cardinal{\{b^{I}:\ (a^{I},b^{I})\in R^{I}\}}<n\}$

es. $(\leq2HaFiglio)^{I}$ è l'insieme delle persone un numero di
figli minore o uguale a 2.


\section{Confronti fra logiche }

In modo analogo a quanto si era fatto con le logiche derivate da K,
denotiamo con ALX la logica AL a cui si aggiunge il costrutto X


\subsection{Equivalenza $\alue$ed $\alc$}

Dimostriamo che $AL\mathcal{UE}\equiv AL\mathcal{C}$\\


IP) $\mathcal{UE}$

Ts)$\mathcal{C}$\\


Si vuole mostrate che a partire in$AL\mathcal{UE}$ posso fare tutto
ciò che faccio in $AL\mathcal{C}$, il che si riduce a mostrare che
in $\alue$posso fare negazioni di concetti qualsiasi.

Questo è immediato per i concetti atomici la cui negazione è per definizione
in $AL$

$\neg\top\equiv\bot$, $\neg\bot\equiv\top$

Per negare l'and di due concetti:

$\neg(D\sqcap E)\equiv\neg D\sqcup\neg E$ che sono concetti più semplici
di cui ricorsivamente posso costruire la negazione

$\neg\forall R.C\equiv\exists R.\neg C$

Con queste semplici trasformazioni posso rendere ricorsivamente più
semplice una qualsiasi formula di $\alc$ fino a portarla in una formula
di $\alue$\\
\\
IP) $\mathcal{C}$

Ts)$\mathcal{UE}$\\


Sfruttando De Morgan possiamo scrivere:

$D\sqcup E\equiv\neg(\neg D\sqcap\neg E)$

$\exists R.C\equiv\neg(\forall R.\neg C)$


\subsection{Confronto con logica del prim ordine}

È possibile vedere le logiche AL, ALC, ALN, (deve essere con identità
nel caso di N)
\begin{itemize}
\item A è un concetto: $a(x)$
\item R è un ruolo: $R(x,y)$
\item $a\ $è un individuo: $a$ è una costante
\end{itemize}
Notiamo che ci sono due sole variabili libere e il dominio è fissato
(controllare) la logica del prim ordine è decibile.
\begin{description}
\item [{$\neg C$:}] $\neg C(x)$
\item [{$C\sqcap D\mathbf{:}$}] \foreignlanguage{english}{$C(x)\vee D(x)$}
\item [{$\forall R.C$:}] $\forall y:(R(x,y)\implies C(y)$)
\item [{$\exists R.C:$}] $\exists y:(R(x,y)\implies C(y))$
\item [{$\leq nR$$,$}] la logica deve essere con unità cioè avere un
predicato di uguaglianza E
\item [{$\exists x_{1},\ x_{2},...,\ x_{n+1}(R(x,x_{1})\wedge R(x,x_{2})\wedge\dots R(x,x_{n})$}] $\implies E(x_{1},x_{2})\vee E(x_{1},x_{2})\vee\dots..\underset{1\leq i\leq j\leq n+1}{\bigvee}E(x_{i},x_{j})$
\end{description}

\subsection{Confronto con logica multimodale}

\noindent L'espressività di ALC è la stessa di $K_{n}$ 

\noindent Infatti ogni ruolo $R_{i}$ si può mettere in corrispondenza
con $[R_{i}]c$


\section{Terminologia}
\begin{itemize}
\item Definizione: 
\end{itemize}
Una definizione associa a un concetto atomico un concetto complesso
(non atomico) es. $Parent\equiv Father\sqcup Mather$, il concetto
atomico viene detto anche simbolo nominale
\begin{itemize}
\item Simbolo di base (o nominale):
\end{itemize}
I ruoli e i concetti che appaiono solo nelle parti destre delle definizioni
\begin{itemize}
\item Interpretazione di base:
\end{itemize}
Un'interpretazione di una T-Box che interpreta solo i simboli di base
\begin{itemize}
\item Aciclico:
\end{itemize}
Nessun simbolo nominale usa sé stesso
\begin{itemize}
\item T-Box definitorio:
\end{itemize}
La sua interpretazione si estende in modo unico a una interpreatazione
di tutto il t-box (estesa)

Un T-Box è definitorio se è possibile costruirne uno equivalente ma
aciclico\\


Non sono definitorie alcune T cicliche es.

$Human\equiv Animal\sqcap\forall hasParent.Human$\\


Ma alcune T cicliche sono definitorie es.

$A\equiv B\sqcap\exists R.(A\sqcap\neg A)$ che è equivalente a:

$A\equiv B\sqcap\exists R.(\bot)$ 


\section{Terminologia generalizzata}

A volte è possibile aggiungere nuovi operatori qua descritti.

\textbf{Inclusione}

Si possono usare T-Box del tipo:

$A\sqsubseteq B$ che vale se e solo se $A^{I}\subseteq B^{I}$

Una terminologia T generalizzata (cioè che contiene assiomi di inclusione)
può essere tradotta in una forma T normalizzata sostituendo le forme
del tipo:

$Woman\sqsubseteq Person$ con $Woman\mbox{\ensuremath{\equiv}}\underline{Woman}\sqcap Person$
dove $\underline{Woman}$ è un nuovo simbolo base (che denota qualità
specifiche)

In termini di espressività, le due forme sono equivalenti

Ogni modello di $T$ è anche modello di $\underline{T}$

Qualsiasi interpretazione base di un modello di $T$ è anche interpretazione
base di un modello di $\underline{T}$

Così facendo si ha che: $Woman^{I}\subseteq Person^{I}$ e quindi
$Woman^{I}\equiv Person\cap(\underline{Woman})^{I}$

\textbf{Set}

$Set{a_{1},a_{2},\dots,a_{n}}^{I}=\{a_{1}^{I},a_{2}^{I},\dots,a_{n}^{I}\}$

\textbf{Fills}

$Fills\ r\ c$ sono gli individui che sono in relazione tramite $r$
agli individui identificati da $c$

es. $FILLS\ :\ Child\ Chris$ sono tutti gli individui con figlio
``Chris''

$(Fills\ R:a)^{I}=\{b\in\Delta^{I}\;:\ (a^{I},b^{I})\in R^{I}\}$


\subsection{Note}

$\exists R.C$ si può sostituire con $\exists R.Set(a)$

Posso pensare i T-Box come composti solo di definizioni

Il T-Box può essere tolto del tutto specificando completamente le
definizioni nell'A-Box

Normalmente nelle Logiche Descrittive si fa l'UNA: l'unicity name
assunction cioè si assume che se si usano due nomi come a, b, allora
$a^{I}\neq b^{I}$


\section{Servizi di Reasoning}

I servizi di reasoning si propongono di risolvono quattro tipi di
problemi

\textbf{Soddifisfacibilità: }Un concetto C si dice soddisfacibile
rispetto al T-Box T se esiste un modello di T tale che $C^{I}$ non
è vuoto

\textbf{Sussunzione: }Un concetto C è sussunto da un concetto D, in
T se $C^{I}\subseteq D^{I}$ per ogni modello di T

\textbf{Equivalenze: }Due concetti C e D si dicono equivalenti rispetto
a T se $C^{I}=D^{I}$per ogni modello di T

\textbf{Disgiunzione: }Due concetti C e D sono disgiunti tispetto
a T se $C^{I}\cap D^{I}=\emptyset$ per ogni modello di T

Mostriamo che una KB che fornisce il servizio di sussunzione risolve
anche gli altri tre problemi, infatti:\\


$C$ è insoddisfacibile se e solo se $\bot\sqsubseteq C$ 

$C,D$sono equivalenti se $C\sqsubseteq D$ e $D\sqsubseteq C$

$C,D$sono disgiunti se $C\sqcap D\sqsubseteq\bot$


\section{Feauture Logic}

Le logiche descrittive più semplici appartengono alla famiglia della
features logics (FL in breve) e sono ottenute indebolendo la più\'{ }famosa
logica AL inibendo i costruttori di concetto più complessi.

La più semplice di queste è $FL_{0}$ che si ottiene da AL togliendo
$\bot$ e $\exists$

Un concetto in \textbf{forma normale }è del tipo:

$A_{1}\sqcap A_{2}\sqcap\dot{\dots A_{n}\forall R_{1}.C_{1}\sqcap\forall R_{2}.C_{2}\dots\sqcap\forall R_{m}.C_{m}}$

dove $C_{j}$ sono concetti in forma normale e $R_{j}$ sono ruoli
atomici\\


Posso scrivere qualsiasi concetto in questo modo con l'accortezza
di:

togliere i duplicati: $A_{i}\neq A_{j}$

``sciogliere'' i ruoli con intersezioni: 

$\forall R.(C\sqcap D)\equiv\forall R.C\sqcap\forall R.D$

	

Siano C e D due concetti $FL_{0}$ in forma normale: \\


$C\equiv A_{1}\sqcap\dots\sqcap A_{m}\sqcap\forall R1.C1\sqcap\forall Rn.Cn$

$D\equiv B1\sqcap\dots\sqcap Bk\sqcap\forall S1.D1\sqcap\forall Sl.Dl$	\\


allora $C\sqsubseteq D$ se e solo se \\


1. per ogni i, $1\leq i\leq k$ esiste un j, $1\leq j\leq m$ per
il quale $Bi=Aj$.

2. per ogni i, $1\leq i\leq l$ esiste un j, $1\leq j\leq n$ per
il quale $Si=Rj$ e$Cj\sqsubseteq Di$.


\subsection{Varianti}

Ad $AL_{0}$può essere aggiunto il concetto $\bot$, ottenendo $AL_{\bot}$e
la forma normale viene a essere la stessa di prima a cui si aggiunge
l'opportunità per un concetto di essere o$\bot$

Notiamo che $\bot\sqsubseteq C$, per ogni $C$

Ad $AL_{0}$ si può aggiungere la negazione di concetti atomici ottenendo
$AL_{\neg}$

Notiamo che $\bot\equiv A\wedge\neg A$, quindi $AL_{\neg}$contiene
strettamente$AL_{0}$	

In $ALC$ per verificare se $C\sqsubseteq D$ possiamo controllare
che $C\sqcap\neg D$ si insoddisfacibile

Per controllare $C\equiv D$ possiamo quindi testare $C\sqcap\neg D$
e $D\sqcap\neg C$.


\section{A-Box Reasoning}

	I servizi di reasoning di A-Box sono:

\textbf{Consistenza}: Un A-Box è consistente in un modello T-Box T
se c'è un'interpreatzione che sia modello sia di T che di A

\textbf{Instance Checking}: Un individuo è istanza di un concetto
C rispetto a un A-Box A se è membro dell'insieme C. $\vera A{C(a)}$
il che vale se e solo se $A\cup\{\neg C(a)\}$ è insoddisfacibile

\textbf{Instance Retrivial}: Trova tutti gli individui che sono istanze
di una data descrizione

\textbf{Problema di Realizzazione}: Trova il concetto più specifico
a cui un individuo appartiene


\subsection{Variante}

T-Box ed A-Box si possono arricchire con $C\implies D$ cioè: per
ogni individuo $a$ cui è noto $C(a)$ si ha anche $D(a)$.

Notiamo che NON vale $\neg D\implies\neg C$

		
