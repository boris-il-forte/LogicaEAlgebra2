
\chapter{Verso la decidibilità - Logica determinata}


\section{Insieme $\Lambda$ consistente e sue proprietà}

Sia $\Lambda$ una logica (cioè ha tutte le tautologie ed è chiusa
rispetto al Modus Ponens)

$\Gamma$ si dice $\Lambda$-consistente se: $\nonSemW{\Gamma}{\Lambda}{\bot}$,
dove $\bot=A\wedge\neg A$

$\Delta$ si dice $\Lambda$-consistente massimale se per ogni fbf
$a$ $a\in\Delta$ oppure $\neg a\in\Delta$ $ $\\


\textbf{Proprietà:} $ $ 
\begin{enumerate}
\item Se $\teoa$ e $\Gamma\subseteq\Delta$ allora $\Delta\teorema a$.
Ovvero se alcune premesse non mi servono posso comunque metterle per
dedurre una formula 
\item Se $\teorGamma a$ e $\Lambda\subseteq\Lambda'$ allora $\Gamma\vdash_{\Lambda'}a$.
Ovvero quello che posso dedurre in una logica più scarna (es. PL)
lo posso dedurre anche in una più ricca che la contien (es. Modale) 
\item se $a\in\Gamma$ allora $\teoa$ . \\
 Infatti $\teorema{a\implies a}$ è un teorema dato che $a\implies a$
è una tautologia 
\item $\{a|\teoa\}$ è la minima logica che contiene $\Gamma\cup\Lambda$.
Infatti posso dedurre tutte le tautologie da $\Gamma$, anche se non
userò nessuna formula di $\Gamma$ ma solo quelle che già sono nella
logica $\Lambda$ $ $ 
\item Se $\teoa$ e $\{a\}$$\teorema b$ allora $\teorGamma b$ \\
 Infatti: per dedurre $a$ uso regole di inferenza, formule di $\Gamma$,
assiomi di $\Lambda$. Per arrivare in $b$ uso assiomi di $\Lambda$
e regole di inferenza, quindi posso arrivare da $\Gamma$ direttamente
in $b$ usando formule di $\Gamma$, regole di inf. e assiomi di $\Lambda$ 
\item Se $\teoa$ e $\teorGamma{\implica ab}$ allora $\teorGamma b$, dato
che $\Lambda$ è chiusa rispetto al MP 
\item $\Gamma\cup\{a\}\teorema b$ se e solo se $\teorGamma{\implica ab}$
\\
 \textbf{Andata}: $\teorema{a_{1}\wedge...\wedge a\wedge...\wedge}a_{n}\implies b$
(per definizione di teorema), si può portare $a$ alla destra dell'implicazione
$\teorema{a_{1}\wedge...\wedge}a_{n}\implies(a\implies b)$ \\
 \textbf{Ritorno}: $\teorema{a_{1}\wedge}...\wedge a_{n}\implies(a\implies b)$,
basta portare $a$ tra le $ $and. 
\item $\teoa$ se e solo se $\Gamma\cup\{\neg a\}$ non è $\Lambda$-consistente
\\
 \\
 \textbf{Andata}: $\teoa$, $\Gamma\teorema{\neg a}$, posso dedure
$\bot$ che è contro la definizione di $\Lambda$-consistenza\\
 \textbf{Ritorno}: Se$ $$\Gamma\cup\{\neg a\}$ non è $\Lambda$-consistente,
allora $\Gamma\cup\{\neg a\}\teorema{\bot}$ da cui per 7. \\
 $\Gamma\teorema{\neg a\implies\bot}$ (sposto $\neg a$ a destra
e metto l'implica), \\
 Dato che $(\neg a\implies\bot)\implies a$ è una tatutologica, per
MP ottengo\\
 $a$ 
\item $\Gamma$ è $ $$\consist$ se e solo se $\exists\beta:\nonSem{\Gamma}{_{\Lambda}\beta}$
\\
 \textbf{Andata}: Basta prendere $\neg a\wedge a$\\
 \textbf{Ritorno}: Se deducessi tutte le formule ($\neg$$\exists\beta:\nonSem{\Gamma}{_{\Lambda}\beta}$
significa $\forall\beta:\teorGamma{\beta}$) , potrei dedurre anche
$\bot$, da cui la non consistenza 
\item $\Gamma$ è $\consist$ se per ogni $a$ \\
 $\Gamma\cup\{a\}$ o $\Gamma\cup\{\neg a\}$ è $\consist$\\
 se $\teoa$ allora $ $$\Gamma\cup\{\neg a\}$ non è consistente
perché con $a$ e $\neg a$ posso dedurre $\bot$, ma $\Gamma\cup\{a\}$
lo è \\
 se $\Gamma\teorema{\neg a}$ allora $ $$\Gamma\cup\{\neg a\}$ è
consistente ma non $\Gamma\cup\{a\}$ 
\item $\bot$$\notin\Gamma$ se $\Gamma$ è $\consist$ (altrimenti potrei
dedurlo per il 3.) 
\item Se $\Delta$è $\consist\: massimale$ e $\Delta\teorema a$ allora
$a\in\Delta$\\
 se $a\notin\Delta$ allora $\neg a\in\Delta$ (dato che $\Delta$è
massimale) \\
 ma se $\Delta$ contiene $\neg a$ allora per il 2.)\\
 $\Delta\teorema{\neg a}$ , che insieme a $\Delta\teorema a$ mi
da $\Delta\teorema{\bot}$ 
\item Se $\Delta$ è $\consMax$ e $ $$a\in\Delta$. $\implica ab\in\Delta$
allora $b\in\Delta$. \\
 Lo si vede subito usando 2.) se tutti e tre, e poi 6.) (deduco $a$,
$\implica ab$, allora deduco anche $b$) 
\end{enumerate}

\section{Insieme $\Lambda$ consistente massimale}

\emph{\large{{{{Lemma di Lindelman - Esistenza dell'insieme $\consMax$}}}}}{\large{{{
}}}}\emph{\large{{{{in una logica $\Lambda$}}}}}{\large{{{
}}}}\emph{\large{{{{consistente}}}}}{\large{{{}}}}\\
 {\large{{{ }}}}\\
 {\large{{{ Considero tutte le formule $b1,\ b2,\ b3,\dots$ della
logica $\Lambda$ (posso farlo perché sono un'infinità numerabile)}}}}{\large \par}

Chiamo $\Gamma_{0}$ un insieme che contiene una sola formula (ad
esempio una tautologia)

Dopodichè iterativamente, per ogni formula mi chiedo\\


$\Gamma_{0}$$\teorema{b1}$ ? $\begin{cases}
si: & \Gamma_{1}=\Gamma_{0}\cup b1\\
no: & \Gamma_{1}=\Gamma_{0}\cup\neg b1
\end{cases}$\\


$\Gamma_{1}$$\teorema{b2}$ ? $\begin{cases}
si: & \Gamma_{2}=\Gamma_{1}\cup b2\\
no: & \Gamma_{2}=\Gamma_{1}\cup\neg b2
\end{cases}$ 
\begin{description}
\item [{$\Delta=\bigcup_{n\geq0}\Gamma_{i}$}] (nota, questa unione è infinita) 
\item [{$\Delta$}] è consistente massimale infatti:\end{description}
\begin{enumerate}
\item Massimale in quanto contiene $a$ oppure $\neg a$ per costruzione 
\item Consistente. Per assurdo se non lo fosse avrei: $\Delta\teorema{\bot}$\\
 cioè esiste un numero finito di formule di $\Delta$ da cui deduco
il falso,\\
 dato che è un numero finito di formule, sta in $\Gamma_{i}$ , cioè
esiste un $\Gamma_{i}$ non consistente, assurdo perché lo sono tutti
per costruzione \lightning 
\end{enumerate}
\emph{\large{{{Nota:}}}}{\large \par}
\begin{itemize}
\item Non sappiamo costruire $\Delta$ perché nasce da unione infinita 
\item Non è unico, infatti se considero formule in ordine diverse potrei
``dire'' si o no in modo diverso \\
 es. $a,\ \implica ab,\ b$ (allora $\Delta$ contiene $b$)\\
 es. $b,\ c$ (allora $\Delta$ contiene $\neg b$) 
\end{itemize}

\subsection{Teorema}

$\teoa$ se e solo se $a\in$ a tutti i quei $\Delta$ $\Lambda-consistenti\ massimali$
tali che: $\Gamma\subseteq\Delta$\\


\textbf{Andata:}

$\teoa$, anche $\Delta\teorema a$ per la 1.)

\textbf{Ritorno:}

Per assurdo, se $\nonSem{\Gamma}{_{\Lambda}a}$ allora $\Gamma\cup\{\neg a\}$
è $\consist$ (per la 8.)

da cui per Lindellman esiste $\Delta'$ che contiene $\Gamma\cup\{\neg a\}$
consistente massimale

data la consistenza $\Delta'$ non contiene $a$, il che è contro
l'ipotesi \lightning 


\section{Lemma di Verità}

Sia $M^{\Lambda}(S^{\Lambda},R^{\Lambda},V^{\Lambda})$ il modello
canonico di $\Lambda$

$\veraCanAlfa a$ se e solo se $a\in\alpha$\\


Ip) $\veraCanAlfa a$ 

TS) $a\in\alpha$\\


Dimostrazione per \textbf{induzione} sul numero n dei connettivi della
formula $a$

\ovalbox{n=0} cioè $a$ è del tipo $A$ (lettera enunciativa) da
cui $\veraCA$ se e solo se $\alpha\in V^{\Lambda}(A)$ se e solo
se $A\in\alpha$

\ovalbox{Ipotesi di Induzione} $a$ con n connettivi, può essere
dei seguenti tipi:
\begin{enumerate}
\item $\neg b$
\item $\implica bc$
\item $\boxx b$
\end{enumerate}
\textbf{Caso 1:} $\veraCA$ se e solo se $\veraCanAlfa{\neg b}$ se
e solo se $\nonveraCan{\alpha}b$\\


$b$ ha $n-1$ connettivi (dato che $b$) ne ha $n$, quindi vale
l'ipotesi di induzione da cui:

$b\notin\alpha$, d'altra parte $\alpha$ è $\consMax$ (per come
è definito $S^{\Lambda}$) da cui:

\textbf{$b\notin\alpha$ }se e solo se \textbf{$ $}$\neg b\in\alpha$
cioè se:

$a\in\alpha$ 

\textbf{Caso} 2:$ $ $\veraCA$ se e solo se

Caso 21: $\nonveraCan{\alpha}b$ 

Caso 22: $\veraCan{\alpha}c$\\


\textbf{Caso 21}: $\nonveraCan{\alpha}b$ \\


Il numero di connettivi di $b$ e di $c$ sommati dà $n-1$

quindi per ipotesi induttiva $\nonveraCan{\alpha}b$ se e solo se
$b\notin\alpha$ 

se e solo se $\neg b\in\alpha$ (per la compattezza max di $\Lambda$)
\textbf{({*})}

D'altra parte $\neg b\implies(b\implies c)$ è una tautologi della
PL e quindi è un teorema di $\Lambda$ (perché un logica contiene
tutte le tautologie)

e quindi $\neg b\implies(b\implies c)$ $\in\alpha$ \textbf{({*}{*})}

da cui per MP con \textbf{({*})} e \textbf{({*}{*})} si ha che $b\implies c$
appartiene ad $\alpha$\\
\\
\textbf{Caso 22}: $\veraCan{\alpha}c$\\


Vale l'ipotesi di induzione da cui:

quindi per ipotesi induttiva $\veraCan{\alpha}c$ se e solo se $c\in\alpha$
\textbf{({*})}

D'altra parte $c\implies(b\implies c)$ è una tautologi della PL e
quindi è un teorema di $\Lambda$ (perché un logica contiene tutte
le tautologie)

e quindi $c\implies(b\implies c)$ $\in\alpha$ \textbf{({*}{*})}

MP \textbf{({*}) }e \textbf{({*}{*}) }ci dà $b\implies c$ appartiene
ad $\alpha$\\
\\
\textbf{Caso 3}: $a$ è del tipo $\boxx b$

Ip)$\veraCan{\alpha}{\boxx b}$

Ts)$\boxx b\in\alpha$\\


Dall'ipotesi segue che $\forall\beta:(\alpha,\beta)\in R^{\Lambda}$
si ha: $\veraCan{\beta}b$ (questo per la definizione di $\boa$)

$b$ ha $n-1$ connettivi quindi vale per lei l'ipotesi di induzione: 

$b\in\beta$

\ovalbox{\parbox[t][1\totalheight][c]{0.9\textwidth}{%
$(\alpha,\beta)\in R^{\Lambda}$ se e solo se: $\{a\ |\ \boa\in\alpha\}\subseteq\beta$

$\alpha\in V^{\Lambda}(A)$ se e solo se: $A\in\alpha$%
}} \\


Ognuno dei $\beta$ con cui $\alpha$ è in relazione è $\consMax$
e ognuno contiene l'insieme $\{a\ |\ \boa\in\alpha\}$ 

$\Gamma\teorema a$ se e solo se $a$ appartiene a tutti i $\Delta_{i}$
$\consMax$ con $\Gamma\subseteq\Delta_{i}$

$\beta\teorema b$ se e solo se $b$ appartiene a tutti i $\Delta_{i}$
$\consMax$ con $\beta\subseteq\Delta_{i}$

$\{a\ |\ \boa\in\alpha\}$ è consistente massimale (davvero??) e quindi

$\{a\ |\ \boa\in\alpha\}$ $\teorema b$, per la 2. definizione equivalente
di Logica Normale ``aggiungo $\square$ ad entrambi i lati'' da
cui:

$\{\boa\ |\ \boa\in\alpha\}$ $\teorema b$

Ma $\{\boa\ |\ \boa\in\alpha\}$ è un sottoinsieme di formule di $\alpha$
quindi a maggior ragione ricavo $b$ da tutto $\alpha$ da cui:

$\alpha\teorema b$\\
Ip) $\boxx b\in\alpha$

TS) $\veraCanAlfa{\boxx b}$\\
Se $\boxx b\in\alpha$ per definizione di $R^{\Lambda}$ per ogni
mondo $\beta$ con $(\alpha,\beta)\in R^{\Lambda}$ si ha $b\in\beta$

Notiamo che $b$ ha $n-1$ connettivi, quindi vale l'ipotesi di induzione
e quindi:

$ $$b\in\beta$ se e solo se $\veraCan{\beta}b$

Dato che questo vale $ $per ogni $\beta$ in relazione con $\alpha$,
si ha: $\veraCanAlfa{\boxx b}$


\section{$ $Correttezza e completezza della logica K rispetto a tutti i Frame}

Dimostriamo che la logica K (minima logica modale normale) è corretta
e completa

Ip) $\teolm Ka$ 

Ts)$\vera Fa$\\


A1, A2, A3 sono tautologie e quindi valide su tutti i frame

MP, sia la regola di necessitazione (RN) fanno passare da formule
valide su un frame a formule valide su quello stesso frame. 

Essendo $a$ l’ultima formula di una sequenza finita di fbf che o
sono istanze degli assiomi A1, A2, A3, K o sono ottenute da fbf precedenti
tramite MP o RN, è una fbf valida su ogni frame.\\
\\
Ip) $\vera Fa$

Ts)$\teolm Ka$ \\
\\
Supponiamo $\nonTeor Ka$, allora per il corollario del lemma di verità\\


\ovalbox{\parbox[t][1\totalheight][c]{0.9\textwidth}{%
Per ogni formula $a$, sia $\Lambda$ una logica, si ha $\vera{M^{\Lambda}}a$
se e solo se $\teorema a$, dove $M^{\Lambda}$ è il modello canonico%
}} \\


Si avrebbe: $\nonvera{M^{K}}a$ da cui anche

$\nonvera{F^{K}}a$, cioè $a$ non valida sul frame su cui $M^{K}$
è costruito quindi:

$\nonvera Fa$ (infatti esiste almeno un frame, $F^{K},$ in cui non
è valida $a$) il che però è contro l'ipotesi \lightning.


\section{$ $Correttezza e completezza della logica K4 rispetto ai Frame transitivi}

Nota: K4 è costruita a partire dalla logica K a cui si aggiunge l'assioma
della transitività $\boa\implies\boxx{\boa}$ \\
Ip)$\teolm{K4}a$ 

Ts) $\entail Fa$ con F frame transitivo\\
\\
Simile al caso precedente in cui anche 4 è valido in quando il frame
è transitivo\\


Ip) $\entail Fa$ con F frame transitivo (cioè la cui relazione è
transitiva)

Ts)$\teolm{K4}a$\\
\\
Per procedere con una dimostrazione sulla falsa riga della precedente
abbiamo bisogno di dimostrare la transitività di $R^{K4}$

cioè della relazione $R^{K4}$ del modello canonico $M^{K4}=(S^{K4},R^{K4},V^{K4})$,
servirà ragionando per assurdo.\\


\textbf{Transitività di $R^{K4}$ }

se $(\alpha,\beta)\in R^{K4}$, $(\beta,\gamma)\in R^{K4}$ allora
$(\alpha,\gamma)\in R^{K4}$

cioè deve avvenire che: $\{a\ |\ \boa\in\alpha\}\subseteq\gamma$
(definizione di essere in relazione $\alpha R^{K4}\gamma$ del modello
canonico)

se $\boa\in\alpha$ , ricordando che:

$\alpha$ è un insieme $K4-consistente\ massimale$,

4 è un teorema della logica 

i teoremi di na logica appartengono a tutti gli insiemi consistenti
massimali rispetto a quella logica quindi

$\alpha$, così come ogni insieme $K4-consistente\ massimale$, contiene
anche $\boa\implies\boxx{\boa}$ (4) da cui

\selectlanguage{english}%
$\boxx{\boa}\in\alpha$\foreignlanguage{italian}{.}

$\{a\ |\ \boa\in$$\alpha\}\subseteq\beta$\foreignlanguage{italian}{
(per ipotesi $\alpha R^{K4}\beta$)}

\selectlanguage{italian}%
ma per ogni formula $\boa\in\alpha$ si ha che $\boxx{\boxx a}\in\alpha$
e quindi: $\{\boa\ |\ \boxx{\boxx a}\in\alpha\}\subseteq\beta$ da
cui $\{\boa\ |\ \boxx a\in\alpha\}\subseteq\beta$

\selectlanguage{english}%
$\{a\ |\ \boa\in$$\beta\}\subseteq\gamma$\foreignlanguage{italian}{
(per ipotesi $\beta R^{K4}\gamma$), ma le formule del tipo $\boa$
contenute in $\beta$ sono le stesse contenute in $\alpha$, quindi
si ha anche:}

$\{a\ |\ \boa\in$$\alpha\}\subseteq\gamma$\foreignlanguage{italian}{
cioè $(\alpha,\gamma)\in R^{K4}$ }\\
\foreignlanguage{italian}{}\\
\foreignlanguage{italian}{A questo punto possiamo usare in modo profiquo
il corollario del lemma di verità.}

\selectlanguage{italian}%
Dimostriamo la tesi per assurdo: supponiamo che $\nonTeor{K4}a$

allora per il corollario del teorema di verità si avrebbe anche $\nonvera{M{}^{K4}}a$ 

e in particolare si avrebbe $\nonvera{F^{K4}}a$, cioè si avrebbe
un Frame transtivo (infatti $R^{K4}$ è transitiva) nel quale non
è valida $a$

ma ciò contraddice l'ipotesi $\entail Fa$ (con $F$ transitivo) \lightning\\



\section{$ $Teorema di Raggiungibilità}

$(\alpha,\beta)\in R^{\Lambda}$ se e solo se $\relazCAB{\alpha}{\beta}$
se e solo se \foreignlanguage{english}{$\relazCAD{\alpha}{\beta}$}
\\
Ip)$\relazCAB{\alpha}{\beta}$

Ts)$\relazCAD{\alpha}{\beta}$\\


Per assurdo supponiamo che

$b\in\beta$ e che $\diamond b\notin\alpha$, 

\selectlanguage{english}%
$\neg\diamond b\in\alpha$\foreignlanguage{italian}{ (dato che $\alpha$)
è $\consMax$}

\selectlanguage{italian}%
$\boxx{\neg b\in\alpha}$ (equivalenza $\boxx{\neg a}$ $\equiv$$\neg\dia$)
da cui

$\neg b\in\beta$ (definizione di $\boxx x$ e considerato $\alpha R^{\Lambda}\beta$)

il che è assurdo perché $b\in\beta$, e $\beta$ è $\consMax$\lightning\\


Ip)$\relazCAD{\alpha}{\beta}$

Ts)$\relazCAB{\alpha}{\beta}$\\
Per assurdo supponiamo cioè che

$\boa\in\alpha$ e $a\notin\beta$

$\neg a\in\beta$ (dato che $\beta$ è $\consMax$)

$\diamond\neg a\in\alpha$ (infatti $\alpha R^{\Lambda}\beta$ e in
$\beta$ è vera $\neg a$, quindi $a$ ha almeno un successore nel
quale $\neg a$ è vera)

$\neg\boxx{a\in\alpha}$ contro l'ipotesi della consistenza e massimalità
di $\alpha$\lightning\\



