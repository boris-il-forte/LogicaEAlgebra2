\global\long\def\veraw#1#2#3{#1\models_{#2}#3}


\global\long\def\vera#1#2{#1\models#2}


\global\long\def\nonvera#1#2{#1\nvDash#2}


\global\long\def\nonveraw#1#2#3{#1\nvDash_{#2}#3}


\global\long\def\nonSem#1#2{#1\nvdash#2}


\global\long\def\nonSemW#1#2#3{#1\nvdash_{#2}#3}


\global\long\def\verita#1#2{#1\in V(#2)}


\global\long\def\entail#1#2{#1\models#2}


\global\long\def\semantica#1#2#3{#1\vdash_{#2}#3}


\global\long\def\semGen#1#2{#1\vdash#2}


\global\long\def\boxx#1{\square#1}


\global\long\def\diam#1{\diamond#1}


\global\long\def\dia{\diamond a}


\global\long\def\boa{\boxx a}


\global\long\def\forhten#1#2#3{\forall#1#2\Rightarrow#3}


\global\long\def\implica#1#2{#1\Rightarrow#2}


\global\long\def\teorema#1{\vdash_{\Lambda}#1}


\global\long\def\teorGamma#1{\Gamma\vdash_{\Lambda}#1}


\global\long\def\teoa{\teorGamma a}



\chapter{Verso la decidibilità - Logica determinata}


\section{Insieme $\Lambda$ consistente e sue proprietà}

$\Gamma$ si dice $\Lambda$-consistente se: $\nonSemW{\Gamma}{\Lambda}{\perp}$,
dove $\perp=A\wedge\sim A$

$\Delta${ si dice $\Lambda$}-consistente massimale se per ogni fbf
$a${ $a\in\Delta$} oppure $\sim a\in\Delta$

\textbf{Proprietà:}
\begin{enumerate}
\item Se $\teoa$ e $\Gamma\subseteq\Delta$ allora $\Delta\teorema a$.
Ovvero se alcune premesse non mi servono posso comunque metterle per
dedurre una formula
\item Se $\teorGamma a$ e $\Lambda\subseteq\Lambda'$ allora $\Gamma\vdash_{\Lambda'}a$.
Ovvero quello che posso dedurre in una logica più scarna (es. PL)
lo posso dedurre anche in una più ricca che la contien (es. Modale)
\item se $a\in\Gamma$ allora $\teoa$ . \\Infatti $\teorema{a\implies a}$
è un teorema dato che $a\implies a$ è una tautologia
\item $\{a|\teoa\}$ è la minima logica che contiene $\Gamma\cup\Lambda$.
Infatti posso dedurre tutte le tautologie da $\Gamma$, anche se non
userò nessuna formula di $\Gamma$ ma solo quelle che già sono nella
logica $\Lambda$
\item Se $\teoa$ e $\{a\}$$\teorema b$ allora $\teorGamma b$ 
\\
Infatti:
per dedurre $a$ uso regole di inferenza, formule di $\Gamma$, assiomi
di $\Lambda$$ $per arrivare in $b$ uso assiomi di $\Lambda$
\item fgsdfg\end{enumerate}

