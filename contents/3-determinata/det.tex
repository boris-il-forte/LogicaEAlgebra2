\global\long\def\veraw#1#2#3{#1\models_{#2}#3}


\global\long\def\vera#1#2{#1\models#2}


\global\long\def\nonvera#1#2{#1\nvDash#2}


\global\long\def\nonveraw#1#2#3{#1\nvDash_{#2}#3}


\global\long\def\nonSem#1#2{#1\nvdash#2}


\global\long\def\nonSemW#1#2#3{#1\nvdash_{#2}#3}


\global\long\def\verita#1#2{#1\in V(#2)}


\global\long\def\entail#1#2{#1\models#2}


\global\long\def\semantica#1#2#3{#1\vdash_{#2}#3}


\global\long\def\semGen#1#2{#1\vdash#2}


\global\long\def\boxx#1{\square#1}


\global\long\def\diam#1{\diamond#1}


\global\long\def\dia{\diamond a}


\global\long\def\boa{\boxx a}


\global\long\def\noa{\sim}


\global\long\def\forhten#1#2#3{\forall#1#2\Rightarrow#3}


\global\long\def\implica#1#2{#1\Rightarrow#2}


\global\long\def\teorema#1{\vdash_{\Lambda}#1}


\global\long\def\teorGamma#1{\Gamma\vdash_{\Lambda}#1}


\global\long\def\teoa{\teorGamma a}


\global\long\def\consist{\mbox{\ensuremath{\Lambda}}-consistente}


\global\long\def\consMax{\Lambda-consistente\: massimale}



\chapter{Verso la decidibilità - Logica determinata}


\section{Insieme $\Lambda$ consistente e sue proprietà}

Sia $\Lambda$ una logica (cioè ha tutte le tautologie ed è chiusa
rispetto al Modus Ponens)

$\Gamma$ si dice $\Lambda$-consistente se: $\nonSemW{\Gamma}{\Lambda}{\bot}$,
dove $\bot=A\wedge\sim A$

$\Delta$ si dice $\Lambda$-consistente massimale se per ogni fbf
$a$ $a\in\Delta$ oppure $\sim a\in\Delta$ $ $\\




\textbf{Proprietà:}
\begin{enumerate}
\item Se $\teoa$ e $\Gamma\subseteq\Delta$ allora $\Delta\teorema a$.
Ovvero se alcune premesse non mi servono posso comunque metterle per
dedurre una formula
\item Se $\teorGamma a$ e $\Lambda\subseteq\Lambda'$ allora $\Gamma\vdash_{\Lambda'}a$.
Ovvero quello che posso dedurre in una logica più scarna (es. PL)
lo posso dedurre anche in una più ricca che la contien (es. Modale)
\item se $a\in\Gamma$ allora $\teoa$ . \\
Infatti $\teorema{a\implies a}$ è un teorema dato che $a\implies a$
è una tautologia
\item $\{a|\teoa\}$ è la minima logica che contiene $\Gamma\cup\Lambda$.
Infatti posso dedurre tutte le tautologie da $\Gamma$, anche se non
userò nessuna formula di $\Gamma$ ma solo quelle che già sono nella
logica $\Lambda$
\item Se $\teoa$ e $\{a\}$$\teorema b$ allora $\teorGamma b$ \\
Infatti: per dedurre $a$ uso regole di inferenza, formule di $\Gamma$,
assiomi di $\Lambda$. Per arrivare in $b$ uso assiomi di $\Lambda$
e regole di inferenza, quindi posso arrivare da $\Gamma$ direttamente
in $b$ usando formule di $\Gamma$, regole di inf. e assiomi di $\Lambda$
\item Se $\teoa$ e $\teorGamma{\implica ab}$ allora $\teorGamma b$, dato
che $\Lambda$ è chiusa rispetto al MP
\item $\Gamma\cup\{a\}\teorema b$ se e solo se $\teorGamma{\implica ab}$
\\
\textbf{Andata}: $\teorema{a_{1}\wedge...\wedge a\wedge...\wedge}a_{n}\implies b$
(per definizione di teorema), si può portare $a$ alla destra dell'implicazione
$\teorema{a_{1}\wedge...\wedge}a_{n}\implies(a\implies b)$ \\
\textbf{Ritorno}: $\teorema{a_{1}\wedge}...\wedge a_{n}\implies(a\implies b)$,
basta portare $a$ tra le $ $and.
\item $\teoa$ se e solo se $\Gamma\cup\{\sim a\}$ non è $\Lambda$-consistente
\\
\\
\textbf{Andata}: $\teoa$, $\Gamma\teorema{\sim a}$, posso dedure
$\bot$ che è contro la definizione di $\Lambda$-consistenza\\
\textbf{Ritorno}: Se$ $$\Gamma\cup\{\sim a\}$ non è $\Lambda$-consistente,
allora $\Gamma\cup\{\sim a\}\teorema{\bot}$ da cui per 7. \\
$\Gamma\teorema{\sim a\implies\bot}$ (sposto $\sim a$ a destra e
metto l'implica), \\
Dato che $(\sim a\implies\bot)\implies a$ è una tatutologica, per
MP ottengo\\
$a$
\item $\Gamma$ è $ $$\consist$ se e solo se $\exists\beta:\nonSem{\Gamma}{_{\Lambda}\beta}$
\\
\textbf{Andata}: Basta prendere $\sim a\wedge a$\\
\textbf{Ritorno}: Se deducessi tutte le formule ($\sim$$\exists\beta:\nonSem{\Gamma}{_{\Lambda}\beta}$
significa $\forall\beta:\teorGamma{\beta}$) , potrei dedurre anche
$\bot$, da cui la non consistenza
\item $\Gamma$ è $\consist$ se per ogni $a$ \\
$\Gamma\cup\{a\}$ o $\Gamma\cup\{\sim a\}$ è $\consist$\\
se $\teoa$ allora $ $$\Gamma\cup\{\sim a\}$ non è consistente perché
con $a$ e $\sim a$ posso dedurre $\bot$, ma $\Gamma\cup\{a\}$
lo è \\
se $\Gamma\teorema{\sim a}$ allora $ $$\Gamma\cup\{\sim a\}$ è
consistente ma non $\Gamma\cup\{a\}$
\item $\bot$$\notin\Gamma$ se $\Gamma$ è $\consist$ (altrimenti potrei
dedurlo per il 3.)
\item Se $\Delta$è $\consist\: massimale$ e $\Delta\teorema a$ allora
$a\in\Delta$\\
se $a\notin\Delta$ allora $\sim a\in\Delta$ (dato che $\Delta$è
massimale) \\
ma se $\Delta$ contiene $\sim a$ allora per il 2.)\\
$\Delta\teorema{\sim a}$ , che insieme a $\Delta\teorema a$ mi da
$\Delta\teorema{\bot}$
\item Se $\Delta$ è $\consMax$ e $ $$a\in\Delta$. $\implica ab\in\Delta$
allora $b\in\Delta$. \\
Lo si vede subito usando 2.) se tutti e tre, e poi 6.) (deduco $a$,
$\implica ab$, allora deduco anche $b$)
\end{enumerate}

\section{Insieme $\Lambda$ consistente massimale}

\emph{\large{Esistenza dell'insieme $\consMax$}} \emph{\large{in
una logica $\Lambda$}} \emph{\large{consistente}}\\
\\
Considero tutte le formule $b1,\ b2,\ b3,\dots$ della logica $\Lambda$
(posso farlo perché sono un'infinità numerabile)

Chiamo $\Gamma_{0}$ un insieme che contiene una sola formula (ad
esempio una tautologia)

Dopodichè iterativamente, per ogni formula mi chiedo\\


$\Gamma_{0}$$\teorema{b1}$ ? $\begin{cases}
si: & \Gamma_{1}=\Gamma_{0}\cup b1\\
no: & \Gamma_{1}=\Gamma_{0}\cup\sim b1
\end{cases}$\\


$\Gamma_{1}$$\teorema{b2}$ ? $\begin{cases}
si: & \Gamma_{2}=\Gamma_{1}\cup b2\\
no: & \Gamma_{2}=\Gamma_{1}\cup\sim b2
\end{cases}$
\begin{description}
\item [{$\Delta=\bigcup_{n\geq0}\Gamma_{i}$}] (nota, questa unione è infinita)
\item [{$\Delta$}] è consistente massimale infatti:\end{description}
\begin{enumerate}
\item Massimale in quanto contiene $a$ oppure $\sim a$ per costruzione
\item Consistente. Per assurdo se non lo fosse avrei: $\Delta\teorema{\bot}$\\
cioè esiste un numero finito di formule di $\Delta$ da cui deduco
il falso,\\
dato che è un numero finito di formule, sta in $\Gamma_{i}$ , cioè
esiste un $\Gamma_{i}$ non consistente, assurdo perché lo sono tutti
per costruzione \lightning \end{enumerate}

