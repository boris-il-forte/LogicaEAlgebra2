
\chapter*{Introduzione}

\addcontentsline{toc}{chapter}{Introduzione}

\label{cap:introduction}



Se voi signorine finirete questo corso, e se sopravviverete, sarete
dispensatori di fbf, pregherete per fare model checking di sistemi
assurdi con automi di Büchi ancora più assurdi, utilizzerete il teorema
di Gödel per sconfiggere la logica inconsistente di satana e di altri
esseri malvagi, esprimerete concetti che non si possono esprimere
nella logica del prim'ordine. Ma fino a quel giorno non siete altro
che buoni a nulla convinti che tutti i cretesi sono stupidi e forse
mentono, che i barbieri si radono da soli leggendo il catalogo di
tutti cataloghi che non includono sé stessi, che $1\neq0.\overline{9}$
e $0.\overline{8}*10\neq8.\overline{8}$, e perderete inutilmente
ore della vostra vita a dimostrare l'ipotesi del continuo con gli
assiomi di Zermelo - Fraenkel.

Lasciate il formaggio fuori dall'aula. 


\chapter{La Logica Modale}


\section{Perchè usare la la logica modale}

La logica proposizionale è una logica corretta, completa e decidibile,
ma è poco espressiva, tutte le formule sono formate da concetti atomici
che possono essere o veri o falsi, e da connettivi logici (di cui
solo due, in realtà, sono necessari).

La logica del prim'ordine invece è estremamente espressiva, non ci
limitiamo solo a concetti atomici ma a predicati che possono essere
veri o falsi rispetto a delle variabili che possiamo quantificare
esistenzialmente e universalmente. Tuttavia la logica del prim'ordine
non è decidibile, e quindi non può essere usata per molte applicazioni
pratiche, che richiedono una risposta certa riguardo la validità di
una determinata formula.

Tra le logiche proposizionali e quelle del prim'ordine, si trovano
le logiche modali, che sono logiche più espressive delle logiche proposizionali,
ma restano decidibile.

Per avere maggiore espressività basta aggiungere due nuovi simboli:
\begin{itemize}
\item l'operatore box ``$\square$'' che verrà letto d'ora in poi come
necessariamente
\item l'operatore diamond ``$\diamond$'' che verrà letto d'ora in poi
come possibilmente
\end{itemize}

\section{Sintassi delle logiche modali}


\subsubsection*{Alfabeto}

Le logiche modali sono definite sul seguente alfabeto:
\begin{itemize}
\item A, B lettere proposizionali , chiamiamo l'insieme delle letetre proposizionali
$\phi$
\item $\neg,\,\wedge,\,\vee,\,\implies,\,\iff$
\item $\square,\,\diamond$
\item ), (
\end{itemize}
L'unica differenza rispetto alla logica modale sono i due connettivi
modali box e diamond.


\subsubsection*{Formule ben formate}

Le formule ben formate sono:
\begin{itemize}
\item Le lettere enunciative 
\item se a è una formula ben formata lo sono anche $\neg a,\,\boa,\,\dia$
\item se a e b sono formule ben formate lo sono anche $a\vee b,\, a\wedge b,\, a\implies b,\, a\iff b$
\item nient'altro è una formula ben formata
\end{itemize}

\subsubsection*{Priorità dei connettivi}

i connettivi hanno la seguente priorità:
\begin{enumerate}
\item $\neg,\,\square,\,\diamond$
\item $\wedge$
\item $\vee$
\item $\implies$
\item $\iff$
\end{enumerate}
Dove 1 è la priorità maggiore (viene applicato prima) e 5 è la priorità
minore. I connettivi con stessa priorità vengono applicati nell'ordine
in cui si trovano.

Le parentesi possono essere usate per cambiare le priorità.


\subsubsection*{Sottoformule}

Sia a una formula qualsiasi

sottoformule(a) è l'insieme così definito:
\begin{itemize}
\item Se $a\in\phi$ allora $sottoformule(a)\equiv\{a\}$
\item Se a è una formula del tipo $\neg b,\,\boxx b,\,\diam b$ allora $sottoformule(a)\equiv\{a\}\,\cup\, sottoformule(b)$
\item Se ha è una formula del tipo $a\vee b,\, a\wedge b,\, a\implies b,\, a\iff b$
allora $sottoformule(a)\equiv\{a\}\,\cup\, sottoformule(b)\,\cup\, sottoformule(c)$
\end{itemize}
Si può costruire un albero con radice la formula a e come rami le
sue sottoformule dirette. questo albero si chiama albero di struttura
della formula.


\section{Semantica di Kripke}


\subsection{La semantica dei mondi possibili}

Lan semantica di Kripke, detta anche dei mondi possibili, considera
che le lettere proprosizionali possono essere vere o false a seconda
del mondo in cui vengono valutate. Questi mondi sono poi collegati
l'un l'altro con una relazione di raggiungibilità.

abbiamo quindi:
\begin{itemize}
\item S insieme dei mondi possibili, detti anche stati
\item $R\subseteq S\times S$ relazione di raggiungibilità tra gli stati
\item $F=(S,\, R)$ frame, un grafo orientati che ha come nodi i mondi e
come archi le possibili transizioni tra un mondo e l'altro, ossia
tra due mondi c'è un arco se e solo se il secondo mondo è raggiungibile
dal primo.
\item $V:\,\phi\longrightarrow\mathcal{P}(S)$ funzione di valutazione,
che ad ogni lettera proposizionale, associa gli stati in cui è vera
\item $\mu=(S,\, R,\, V)$ modello, ossia un frame con una funzione di valutazione
per le lettere proposizionali.
\end{itemize}

\subsection{Semantica delle logiche modali:}

Diciamo che $a$ è vera nel mondo $\alpha$ del modello $\mu$, e
scriviamo $\mu\models_{\alpha}a$, se:
\begin{itemize}
\item $\mu\models_{\alpha}A\,\implies\verita{\alpha}A$ con $A\in\phi$ 
\item $\veraw{\mu}{\alpha}{b\vee c}\iff\veraw{\mu}{\alpha}b\,\vee\,\veraw{\mu}{\alpha}c$
\item $\veraw{\mu}{\alpha}{b\wedge c}\iff\veraw{\mu}{\alpha}b\,\wedge\,\veraw{\mu}{\alpha}c$
\item $\veraw{\mu}{\alpha}{(b\implies c)}\iff\nonveraw{\mu}{\alpha}b\,\vee\,\veraw{\mu}{\alpha}c$
\item $\veraw{\mu}{\alpha}{(b\iff c)}\iff(\nonveraw{\mu}{\alpha}b\,\wedge\,\nonveraw{\mu}{\alpha}c)\,\vee\,(\veraw{\mu}{\alpha}c\,\wedge\,\veraw{\mu}{\alpha}c)$
\item $\veraw{\mu}{\alpha}{\boxx b\iff\forall\beta\in S\,:\,(\alpha,\,\beta)\in R\,\veraw{\mu}{\beta}b}$
\item $\veraw{\mu}{\alpha}{\diam b\iff\exists\beta\in S\,:\,(\alpha,\,\beta)\in R\,\veraw{\mu}{\beta}b}$
\end{itemize}
Diciamo che a è vera nel modello $\mu$ se a è vera in ogni mondo
del modello $\mu$, e scriviamo:

$\vera{\mu}a$ 

Diciamo che a è valida su un frame F se è vera in tutti i modelli
costruibili sul frame F, e scriviamo:

$\vera Fa$

Chiamiamo fuynzione di valutazione associata al mondo $V_{\alpha}(A)$
che è definita così:

$V_{\alpha}(A)=0\iff\alpha\notin V(A)$

Tutte le formule che cominciano coi connettivi modali sono dette formule
semi atomiche.

Le tautologie nella logica modale sono quelle formule che sono vere
per tutte le interpretazioni delle formule atomiche e semi atomiche.
