\global\long\def\veraw#1#2#3{#1\models_{#2}#3}


\global\long\def\vera#1#2{#1\models#2}


\global\long\def\verita#1#2{#1\in V(#2)}


\global\long\def\entail#1#2{#1\models#2}


\global\long\def\semantica#1#2#3{#1\vdash_{#2}#3}


\global\long\def\semGen#1#2{#1\vdash#2}


\global\long\def\boxx#1{\square#1}


\global\long\def\diam#1{\diamond#1}


\global\long\def\dia{\diamond a}


\global\long\def\boa{\boxx a}


\global\long\def\forhten#1#2#3{\forall#1#2\Rightarrow#3}


\global\long\def\implica#1#2{#1\Rightarrow#2}



\chapter{Semantica}


\section{Semantica}

$\semGen ab$ cioè a è conseguenza semantica di b, se in ogni Frame,
Modello e Mondo in cui $\vera{\mu}b$ si ha anche $\vera{\mu}a$\\


$\dia\equiv\sim\boxx{\sim a}$

Vale da sinistra a destra,

Infatti:

se $\veraw{\mu}{\alpha}{\dia}$ allora

$\exists\beta:$$\alpha R\beta$ e $\veraw{\mu}{\beta}a$ da cui:

$\mu\nvDash_{\beta}\sim a$

per questo in $\alpha$ non vale $\boxx{\sim a}$ (perché non vale
$\sim a$ in $\beta$)

allora in $\alpha$ vale $\sim\boxx{\sim a}$ cioè $\veraw{\mu}{\alpha}{\sim\boxx{\sim a}}$
cioè la tesi. \\


Vale anche da destra a sinistra, dimostrazione simile. 
