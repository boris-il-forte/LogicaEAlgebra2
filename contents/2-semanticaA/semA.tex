\global\long\def\veraw#1#2#3{#1\models_{#2}#3}


\global\long\def\vera#1#2{#1\models#2}


\global\long\def\nonvera#1#2{#1\nvDash#2}


\global\long\def\nonveraw#1#2#3{#1\nvDash_{#2}#3}


\global\long\def\nonSem#1#2{#1\nvdash#2}


\global\long\def\nonSemW#1#2#3{#1\nvdash_{#2}#3}


\global\long\def\verita#1#2{#1\in V(#2)}


\global\long\def\entail#1#2{#1\models#2}


\global\long\def\semantica#1#2#3{#1\vdash_{#2}#3}


\global\long\def\semGen#1#2{#1\vdash#2}


\global\long\def\boxx#1{\square#1}


\global\long\def\diam#1{\diamond#1}


\global\long\def\dia{\diamond a}


\global\long\def\boa{\boxx a}


\global\long\def\forhten#1#2#3{\forall#1#2\Rightarrow#3}


\global\long\def\implica#1#2{#1\Rightarrow#2}


\global\long\def\teorema#1{\vdash_{\Lambda}#1}



\chapter{Semantica}


\section{Simboli secessari}

$\semGen ab$ cioè a è conseguenza semantica di b, se in ogni Frame,
Modello e Mondo in cui $\vera{\mu}b$ si ha anche $\vera{\mu}a$\\


$\dia\equiv\sim\boxx{\sim a}$

Vale da sinistra a destra,

Infatti:

se $\veraw{\mu}{\alpha}{\dia}$ allora

$\exists\beta:$$\alpha R\beta$ e $\veraw{\mu}{\beta}a$ da cui:

$\mu\nvDash_{\beta}\sim a$

per questo in $\alpha$ non vale $\boxx{\sim a}$ (perché non vale
$\sim a$ in $\beta$)

allora in $\alpha$ vale $\sim\boxx{\sim a}$ cioè $\veraw{\mu}{\alpha}{\sim\boxx{\sim a}}$
cioè la tesi. \\


Vale anche da destra a sinistra, dimostrazione simile. \\



\section{Logiche}

Una logica $\Lambda$ su L è un insieme di fbf su L che: 
\begin{itemize}
\item contiene tutte le tautologie 
\item è chiusa rispetto al Modus Ponens 
\end{itemize}
Ad esempio; $PL(\phi)$ cioè i teoremi della logica proposizionale

Altro esempio $\Lambda_{C}=\{a\,|\,\vera F{a\: per}\ ogni\ F\in C\}$

infatti: 
\begin{itemize}
\item contiene tutte le tautologie perché sono vere mondo per mondo dappertutto 
\item MP : suppongo che in un mondo $\alpha$ accada che: $\nonveraw{\mu}{\alpha}b$
, $\veraw{\mu}{\alpha}a$ . Se vale anche $\veraw{\mu}{\alpha}{\implica ab}$
... l'antecedente è vero, quindi dato che l'implicazione è vera, deve
essere vero anche il conseguente da cui non può che essere $\veraw{\mu}{\alpha}b$ 
\end{itemize}
Una logica si dice \textbf{uniforme }se è chiusa rispetto a sostituzioni
uniformi cioè se sostituendo a una lettere uguali formule uguali in
una tautologia, ottengo una tautologia.

Es. $\Lambda_{C}=\{a\,|\,\vera F{a\: per}\ ogni\ F\in C\}$ NON è
uniforme infatti se considero $V(A)=S$, dove S sono tutti gli stati
possibili (mondi), vale anche $\veraw{\mu}{\alpha}A$, e cioè A è
una tautologia, se al posto di A sostituisco $B\wedge\sim B$ (falsa
in ogni modello e mondo) non ottengo una tautologia.\\
\\


\textbf{\emph{\large{{Teorema}}}}{\large \par}

Sono equivalenti: 
\begin{enumerate}
\item $\Lambda$ è normale 
\item per ogni intero n $\geq0$,


$\teorema{a1\wedge a2\wedge...\wedge an}\implies a$ implica $\teorema{\boa1\wedge\boa2\wedge...\wedge\boa n}\implies\boa$

\item valgono:

\begin{enumerate}
\item $\teorema{\boxx T}$ 
\item $\teorema{\boa\wedge\boxx b}\implies\boxx{(a\wedge b)}$ 
\item $\teorema{\implica ab}$ implica $\teorema{\boa\implies\boxx b}$ 
\end{enumerate}
\end{enumerate}
Dimostrazione

$1\implies$2

per induzione.

se n = 0 allora $\teorema a$ allora $\teorema{\boa}$ per la regola
RN che vale in $\Lambda$ per ipotesi

se n > 0 (passo induttivo) suppongo valga l'antecedente, altrimenti
2 vale senz'altro;

Ricordiamo che $a1\wedge a2\wedge...\wedge an\implies a\equiv a1\wedge a2\wedge...a_{n-1}\implies(an\implies a)$ 
