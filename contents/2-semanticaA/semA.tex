
\chapter{Semantica}


\section{Formule equivalenti}

$a\equiv b$ cioè a è semanticamente equivalente a b, se:
\begin{itemize}
\item $\forall F\,\vera Fb\iff\vera Fa$
\item $\forall\mu\,\vera{\mu}b\iff\vera{\mu}a$
\item $\forall s\in S\,\veraw{\mu}sa\iff\veraw{\mu}sb$
\end{itemize}
si può anche dire che due formule sono semanticamente equivalenti
se:

$\vera ab\,\vee\,\vera ba$

Oppure infine se è valida in ogni frame:

$\valida{a\iff b}$


\section{Connettivi minimi}

Per ogni formula possiamo scriverne una equivalente che usa solo tre
connettivi: $\neg,\,\implies,\,\square$.

Infatti come è ben noto tutti i connettivi proposizionali si possono
esprimere in funzione della negazione e dell'implicazione, mentre
per quanto riguarda il connettivo diamond:

$\dia\equiv\neg\boxx{\neg a}$

Infatti:

se $\veraw{\mu}{\alpha}{\dia}$ allora

$\exists\beta:$$\alpha R\beta$ e $\veraw{\mu}{\beta}a$ da cui:

$\mu\nvDash_{\beta}\neg a$

per questo in $\alpha$ non vale $\boxx{\neg a}$ (perché non vale
$\neg a$ in $\beta$)

allora in $\alpha$ vale $\neg\boxx{\neg a}$ cioè $\veraw{\mu}{\alpha}{\neg\boxx{\neg a}}$
cioè la tesi. 

Similmente si dimostra l'altro verso dell'equivalenza.


\section{Logiche}


\subsection{Logica $\Lambda$}

Una logica $\Lambda$ su L è un insieme di fbf su L che: 
\begin{itemize}
\item contiene tutte le tautologie 
\item è chiusa rispetto al Modus Ponens 
\end{itemize}
Ad esempio; $PL(\phi)$ cioè i teoremi della logica proposizionale

Altro esempio $\Lambda_{C}=\{a\,|\,\vera F{a\: per}\ ogni\ F\in C\}$

infatti: 
\begin{itemize}
\item contiene tutte le tautologie perché sono vere mondo per mondo dappertutto 
\item MP : suppongo che in un mondo $\alpha$ accada che: $\nonveraw{\mu}{\alpha}b$
, $\veraw{\mu}{\alpha}a$ . Se vale anche $\veraw{\mu}{\alpha}{\implica ab}$
... l'antecedente è vero, quindi dato che l'implicazione è vera, deve
essere vero anche il conseguente da cui non può che essere $\veraw{\mu}{\alpha}b$ 
\end{itemize}
Una logica si dice \textbf{uniforme }se è chiusa rispetto a sostituzioni
uniformi cioè se sostituendo a una lettere uguali formule uguali in
una tautologia, ottengo una tautologia.

Es. $\Lambda_{C}=\{a\,|\,\vera F{a\: per}\ ogni\ F\in C\}$ NON è
uniforme infatti se considero $V(A)=S$, dove S sono tutti gli stati
possibili (mondi), vale anche $\veraw{\mu}{\alpha}A$, e cioè A è
una tautologia, se al posto di A sostituisco $B\wedge\neg B$ (falsa
in ogni modello e mondo) non ottengo una tautologia.


\subsection{Logiche modali normali}

Le logiche normali sono logiche che contengono lo schema K

$K:\,\boxx{(a\implies b)}\implies(\boa\implies\boxx b)$

Ed sono chiuse rispetto alla regola di necessitazione:

$RN:\,\dfrac{a}{\boa}$

La logica normale ha i seguenti assiomi:

$A1:\, a\implies(b\implies a)$

$A2:\,(a\implies(b\implies c))\implies((a\implies b)\implies(a\implies c))$

$A3:\,(\neg a\implies\neg b)\implies((\neg a\implies b)\implies a)$

$K:\,\boxx{(a\implies b)}\implies(\boa\implies\boxx b)$

$MP:\,\dfrac{a,\, a\implies b}{b}$

$RN:\,\dfrac{a}{\boa}$

L'intersezione di tutte le logiche normali, è una logica normale (ed
è la minima ) che non ha altri assiomi.

I teoremi sono le ultime formule della dimostrazione, ossia le formule
che ottengo dopo un numero finito di applicazione degli assiomi oppure
utilizzando la regola di necessiazione o il modus ponens.

La minima logica normale viene chiamata logica K.


\subsection{Teorema}

Sono equivalenti: 
\begin{enumerate}
\item $\Lambda$ è normale 
\item per ogni intero n $\geq0$,


$\teorema{a1\wedge a2\wedge...\wedge an}\implies a$ implica $\teorema{\boa1\wedge\boa2\wedge...\wedge\boa n}\implies\boa$

\item valgono:

\begin{enumerate}
\item $\teorema{\boxx T}$ 
\item $\teorema{\boa\wedge\boxx b}\implies\boxx{(a\wedge b)}$ 
\item $\teorema{\implica ab}$ implica $\teorema{\boa\implies\boxx b}$ 
\end{enumerate}
\end{enumerate}
Dimostrazione

$1\implies$2

per induzione.

se n = 0 allora $\teorema a$ allora $\teorema{\boa}$ per la regola
RN che vale in $\Lambda$ per ipotesi

se n > 0 (passo induttivo) suppongo valga l'antecedente, altrimenti
2 vale senz'altro;

si può dimostrare quindi nel seguente modo:

$\teolm{\Lambda}{a_{1}\wedge a_{2}\wedge...\wedge a_{n}n\implies a}$

$\teolm{\Lambda}{a_{1}\wedge a_{2}\wedge...\wedge a_{n-1}\implies(a_{n}\implies a)}$ 

$\teolm{\Lambda}{\boxx{a_{1}}\wedge\boxx{a_{2}}\wedge...\wedge\boxx{a_{n-1}}\implies\boxx{(a_{n}\implies a)}}$
-- per ipotesi di induzione

$\teolm{\Lambda}{\boxx{a_{1}}\wedge\boxx{a_{2}}\wedge...\wedge\boxx{a_{n-1}}\implies(\boxx{a_{n}\implies\boa}})$
-- per K

$\teolm{\Lambda}{\boxx{a_{1}}\wedge\boxx{a_{2}}\wedge...\wedge\boxx{a_{n-1}\wedge\boxx{a_{n}}}\implies\boa}$ 

E la tesi è dimostrata.

2$\implies$1

$\teolm{\Lambda}{(a\wedge(a\implies b))\implies b}$ -- per MP

$\teolm{\Lambda}{(\boa\wedge\boxx{(a\implies b))}}\implies\boxx b$
-- per enunciato 2

$\teolm{\Lambda}{\boxx{(a\implies b)}}\implies\boa\implies\boxx b$
-- che è K

Abbiamo ricavato usando solo il modus ponens e l'enunciato 2, l'assioma
K. segue quindi la tesi.

1$\implies3$

$\teolm{\Lambda}{\top}$

$\teolm{\Lambda}{\boxx{\top}}$-- per RN

$\teolm{\Lambda}{a\wedge b\implies a\wedge b}$ -- per tautologia
($a\implies a)$

$\teolm{\Lambda}{\boa\wedge\boxx b\implies}\boxx{(a\wedge b)}$ --
per proposizione 2

$\teolm{\Lambda}{a\implies b}$ -- per ipotesi

$\teolm{\Lambda}{\boxx{(a\implies b)}}$ -- per RN

$\teolm{\Lambda}{\boxx{(a\implies b)}\implies(\boa\implies\boxx{b)}}$
-- per K

$\teolm{\Lambda}{\boa\implies\boxx b}$ -- per MP

La tesi allora è verificata.

3$\implies$1

dimostriamo due tesi: che la 3 è chiusa rispetto alla necessitazione
e che implica l'assioma K.

$\teolm{\Lambda}a$

$\teolm{\Lambda}a\implies(\top\implies a)$ -- per A1

$\teolm{\Lambda}{\top\implies a}$ -- per MP

$\teolm{\Lambda}{\boxx{\top}\implies\boa}$ -- per 3.c

$\teolm{\Lambda}{\boa}$ -- per 3.a e MP

abbiamo così dimostrato la chiusura secondo la necessitazione.

$\teolm{\Lambda}{a\wedge b\implies c}$

$\teolm{\Lambda}{\boxx{(a\wedge b)\implies\boxx c}}$ -- per 3.c

$\teolm{\Lambda}{\boa}\wedge\boxx b\implies\boxx{(a\wedge b)}$ --
per 3.b

$\teolm{\Lambda}{\boa\wedge\boxx b\implies\boxx c}$ -- per la combinazione
delle due implicazioni precedenti

$\teolm{\Lambda}a\wedge(a\implies b)\implies b$ -- per tautologia

$\teolm{\Lambda}{\boa}\wedge\boxx{(a\implies b)}\implies\boxx b$
-- per applicazione dello schema $\boa\wedge\boxx b\implies\boxx c$
dimostrato precedentemente

$\teolm{\Lambda}{\boxx{(a\implies b)}\implies(\boa\implies\boxx{b)}}$

e così è dimostrato che K è implicato da 3. Il teorema dunque è dimostrato.


\section{Deducibilità di una formula}

Una formula a si dice deducibile da un insieme di formule $\Gamma$
in una logica $\Lambda$ e si scrive:

$\sintattica{\Gamma}{\Lambda}a$

se e solo se:

$\teorema{a_{1}\wedge\,...\,\wedge a_{n}\implies a}$

con $a_{1},\,...\,,\, a_{n}\in\Gamma$

Cioè, una formula a si dice deducibile da un insieme di formule $\Gamma$
se e solo se la congiunzione di formule che formano $\Gamma$ implica
la formula a

si noti che:

$\sintattica{\Gamma}{\Lambda}a\implies\sintattica{\{\boxx b\,|b\in\Gamma\}}{\Lambda}{\boa}$
